\chapter{ATMega32u4}
\section{Register beschreiben}
\textbf{"1" in Register schreiben:}
\begin{lstlisting}[language=C]
// REGISTER = REGISTER | (1 << POSITION IM REGISTER)
// Beispiel:
DDRD = DDRD | (1 << 3);
\end{lstlisting}

\textbf{"0" in Register schreiben:}
\begin{lstlisting}[language=C]
// REGISTER = REGISTER &~ (0 << POSITION IM REGISTER)
// Beispiel:
DDRD = DDRD &~ (1 << 3);
\end{lstlisting}

Es können \textbf{jeweils} mehrere Bits eines Registers in einer Zeile auf 1 \textbf{oder} 0 gesetzt werden. Allerdings darf in einer Zeile ein Bit nicht auf 0, während ein anderes auf 1 gesetzt werden.\\

\textbf{Beispiel}
\begin{itemize}
    \item \underline{Erlaubt:} \\
    \verb|DDRD = DDRD &~ (1 << 3) &~ (1 << 3) &~ (1 << 3);|
    \item \underline{Nicht erlaubt:} \\
    \verb$DDRD = DDRD | (1 << 3) | (1 << 3) &~ (1 << 3);$
\end{itemize}
\todo{Ergibt keinen Sinn, dass das gleiche Bit beschrieben wird oder? Sollte zB (1<<3) und (1<<2) sein.}

\newpage

\section{Takt}
Der Takt des ATMega32u4 kann per Software verringert werden und wird über das \verb|CLKPR|-Register getan. Bevor dieses beschrieben werden kann, muss \verb|0x80| in das Register geschrieben werden.
\todo{Kann Taktgeschwindigkeit nicht auch erhöht werden?}
\todo{Insert Pics}

\textbf{Beispiel}
Externer Takt ist 16MHz und soll auf 8MHz heruntergesetzt werden.
\begin{lstlisting}[language=C]
    CLKPR = 0x80;
    CLKPR = 0x01;
\end{lstlisting}

\section{GPIO}
\todo{Wdym Features?}

\todo{Insert Pics}

\textbf{Beispiel}
Pin-D7 auf HIGH setzen.
\begin{lstlisting}[language=C]
    DDRD = DDRD | (1 << DDD7);
    PORTD = PORTD | (1 << PORTD7);
\end{lstlisting}

\textbf{Wichtig:} Bei der Verwendung von Hardwareeinheiten (Timer, UART, etc.) muss GPIO immer \underline{zuerst} auf Input bzw. Output eingestellt werden.

\section{ADC}
\begin{itemize}
    \item \underline{Single-Ended:} \\
    Spannung von ADC-Pin zu GND wird gemessen.
    \item \underline{Differenziell:} \\
    Spannung zwischen zwei ADC-Pins wird gemessen. (Siehe Kapitel \ref{})
    \item \underline{Referenzspannung:} \\
    Es gibt drei verschiedene Spannungsreferenzen:
    \begin{itemize}
        \item Interne 2,56V Referenz
        \item Externer \verb|AREF|-Pin
        \item Externer \verb|AVCC|-Pin
    \end{itemize}
    \item \underline{Auto-Trigger Mode} \\
    Es wird periodisch gemessen, wofür die Taktquelle eingestellt werden muss. (Siehe Kapitel \ref{Atmega/AutoTrigger})
\end{itemize}

\subsection{Differenziell} \label{Atmega/Differenziell}
Wenn differenziell gemessen wird, ist das Ergebnis im Zweierkomplement dargestellt - ein Zahlensystem um negative Zahlen (in binär) darzustellen.
\textbf{Beispiel} \\
"$-2_d$" im Zweierkomplement \\
\begin{enumerate}
    \item Zunächst wird der Binärwert des Betrags der Zahl invertiert: $2_d=0010_b\Rightarrow 1101_b$ \\
    \item Danach wird zu diesem Wert $1_d$ addiert: $1101_b + 1_b = 1110_b = -2_d$
\end{enumerate}

\subsection{Auto-Trigger Mode} \label{Atmega/AutoTrigger}
Die Taktquelle wird folgendermaßen eingestellt:
\todo[inline]{Insert Pic}

\textbf{Ergebnis} \\
Das Messergebnis des ADC befindet sich in zwei Registern: \verb|ADCL| (ADC-Low) und \verb|ADCH| (ADC-High). Der Messwert kann in zwei Arten dargestellt werden:
\begin{enumerate}
    \item \underline{Linksbündig:}
    \item \underline{Rechtsbündig:}
\end{enumerate}

\verb|ADCL| muss immer vor \verb|ADCH| ausgelesen werden; folgende Beispiele verwenden Linksbündigkeit.\\

\subsection{Messdauer berechnen}
Die ADC Messdauer muss eingestellt werden: kurze Messdauern führen zu ungenaueren Erbenissen, bei zu langen kann die Dauer zwischen Abtastpunkten zu groß werden. Generell sollte die Messfrequenz des ATMega32u4 zwischen $50kHz$ und $200kHz$ sein (wenn die Messgeschwindigkeit realisierbar ist.)

\todo{Insert Pic}

Der Wert des obigen Diagramms muss in die gemessene Spannung umgerechnet werden.

\textbf{Single-Ended}
\begin{align}
    V_{IN} = \frac{ADC \cdot V_{REF}}{1023}
\end{align}

Und der entsprechende Code um die gemessene Spannung zurückzubekommen:
\begin{lstlisting}[language=C]
// SETUP:
DDRF = DDRF &~ (1 << DDF0);	 // PF0-Input
ADMUX = ADMUX | (1 << ADLAR) | (1 << REFS0); // Left adjust ADC result; Voltage reference

// Uncomment for Auto trigger mode; 
// ADCSRA = ADCSRA | (1 << ADATE);
// ADCSRB = ADCSRB | (1 << ADTS1) | (1 << ADTS0); // Auto trigger mode Taktquelle (Timer0)

// Enable Interrupt
// Enable ADC
// Prescaler = 64: ADC_f = 8M/64
ADCSRA = ADCSRA | (1 << ADEN) | (1 << ADSC) | (1 << ADPS2) | (1 << ADPS1);
DIDR0 = DIDR0 | (1 << ADC0D); // Disable digital function of PF0

// READ:
adcRead() {
    uint16_t adc_value;
    float out;
    unsigned char adcl, adch;

    // Start conversation
    // Put in comment in auto trigger mode
    ADCSRA = ADCSRA | (1 << ADSC);

    // Wait for ADC to finish
    while(ADCSRA & (1 << ADSC))  {}

    adcl = ADCL;
    adch = ADCH;
    adc_value = (adcl >> 6) + (adch << 2);
    out = (float)((adc_value * 5) / 1023.0);
    return out;
}
\end{lstlisting}
\todo{No return type?}

\textbf{Differenziell}
\begin{align}
    V_{POS} - V_{NEG} = \frac{ADC \cdot V_{REF}}{GAIN \cdot 512}
\end{align}

Und auch der Code dazu, um die Spannung returniert zu bekommen:
\begin{lstlisting}[language=C]
// SETUP:
DDRF = DDRF &~ (1 << DDF0) &~ (1 << DDF1); // PF0-Input
ADMUX = ADMUX | (1 << ADLAR) | (1 << REFS0); // Left adjust ADC result; Voltage reference
ADMUX = ADMUX | (1 << MUX4); // PINS(P:ADC0; N: ADC1); GAIN: 1

// Uncomment for auto trigger mode
// ADCSRA = ADCSRA | (1 << ADATE);	
// ADCSRB = ADCSRB | (1 << ADTS1) | (1 << ADTS0); // Auto trigger mode Taktquelle (Timer0)

// Enable interrupt enable
// Enable ADC
// Prescaler = 64: ADC_f = 8M/64
ADCSRA = ADCSRA | (1 << ADEN) | (1 << ADSC) | (1 << ADPS2) | (1 << ADPS1);
    
// READ:
adcRead() {
    uint16_t adc_value;
    float out;
    unsigned char adcl, adch;
    
    // Start conversation
    // Put in comment in auto trigger mode
    ADCSRA = ADCSRA | (1 << ADSC);

    // Wait for ADC to finish
    while(ADCSRA & (1 << ADSC)) {}
    adcl = ADCL;
    adch = ADCH;
    adc_value = (adcl >> 6) + (adch << 2);
    out = (float)((adc_value * 5) / (1.0 * 1023.0));
    return out;
}
\end{lstlisting}