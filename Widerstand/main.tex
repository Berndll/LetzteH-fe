\chapter{Widerstand}

\section{Ohm'sches Gesetz}
Der Zusammenhang zwischen Spannung $U$, Strom $I$ und Widerstand $R$:
\begin{align}
    U = R\cdot I \hspace{1cm} I = \frac{U}{R} \hspace{1cm} R = \frac{I}{R}
\end{align}

Der Leitwert $G$ wird in Siemens ($S$) angegeben und ist der Kehrwert des Widerstands:
\begin{align}
    G = \frac{1}{R} = \frac{I}{U}
\end{align}

\section{Netzwerke}
\subsection{Serienschaltung}
\begin{center}
\begin{circuitikz}
    \draw (0,0) to[R=$R_1$] ++(2,0);
    \draw (2,0) to[R=$R_2$] ++(2,0);
\end{circuitikz}
\end{center}

\begin{align}
    R_g &= \sum_{i=0}^{N} R_i = R_1 + R_2 + ...
\end{align}

\subsection{Parallelschaltung}
\begin{center}
\begin{circuitikz}
    \draw (0,0) to[R=$R_1$] ++(2,0);
    \draw (0,-2) to[R=$R_2$] ++(2,0);

    \draw[black] (0,0) -- (0,-2);
    \draw[black] (2,0) -- (2,-2);

    \draw[black] (-1,-1) -- (0,-1);
    \draw[black] (2,-1) -- (3,-1);
\end{circuitikz}
\end{center}

\begin{align}
    \frac{1}{R_g} &= \sum_{i=0}^{N} \frac{1}{R_i} = \frac{1}{R_1} + \frac{1}{R_2} + ...
\end{align}

Bei zwei parallel geschaltenen Widerständen gilt auch:
\begin{align}
    R_g = \frac{R_1 \cdot R_2}{R_1 + R_2}
\end{align}

\subsection{Spannungsteiler}
Die Spannung wird auf zwei Widerstände (bzw. Lasten) aufgeteilt. Über die Kirchhoff'schen Gesetze der Knoten- und Maschenregel können dadurch die einzelnen Spannungen, die auf den Widerständen abfallen berechnet werden.

\subsubsection*{Beispiel}
\begin{center}
\begin{circuitikz}
    \draw (0,0)
        to[R=$R_1$] ++(2,0)
        to[R=$R_2$] ++(2,0);
\end{circuitikz}
\end{center}

Es gilt:
\begin{align}
    U_{R_1} &= U_g \cdot \frac{R_1}{R_1 + R_2} \\
    U_{R_2} &= U_g \cdot \frac{R_2}{R_1 + R_2} \\
\end{align}

Da die Widerstände in Serie geschalten wurden, sind die Einzelströme gleich groß, d.h.:
\begin{align}
    I_{R_1} = I_{R_2}
\end{align}

\section{Leitungswiderstand}
\begin{align}
    R &= \frac{\rho \cdot l}{A} \\
    G &= \frac{A}{\rho \cdot l}
\end{align}
\begin{itemize}
    \item $\rho$ ... materialspezifischer Widerststand in [$\frac{\Omega \cdot mm^2}{m}$] oder [$\frac{mm^2}{S \cdot m}$]
    \item $l$ ... Länge der Leitung in [$m$]
    \item $A$ ... Querschnittsfläche der Leitung in [$m^2$]
\end{itemize}

\section{Sterndreiecktransformation}
Die Sterndreiecktransformation kann verwendet werden, um das Arbeiten mit gewissen Widerstandsnetzwerken zu erleichtern.

% \subsubsection*{Stern}
\begin{center}
\begin{tikzpicture}
    \begin{scope}[xshift=-3cm]    
        \draw (0,0) to[R=$R_2$] ++(330:2);
        \draw (0,0) to[R=$R_1$] ++(210:2);
        \draw (0,0) to[R=$R_3$] ++ (90:2);
    \end{scope}

    \begin{scope}[
            xshift=3cm,
            yshift=-1.25cm
        ]
        \draw  (1.5,0) to[R=$R_a$] ++(120:3.5);
        \draw (-2,0) to[R=$R_b$] ++ (60:3.5);
        \draw (-2,0) to[R=$R_c$] ++  (0:3.5);
    \end{scope}
\end{tikzpicture}
\end{center}

\subsubsection*{Stern-zu-Dreieck}
\begin{align}
    R_1 &= \frac{R_b \cdot R_c}{R_a + R_b + R_c} \\
    R_2 &= \frac{R_a \cdot R_c}{R_a + R_b + R_c} \\
    R_3 &= \frac{R_a \cdot R_b}{R_a + R_b + R_c}
\end{align}

\subsubsection*{Dreieck-zu-Stern}
\begin{align}
    R_a &= \frac{R_1 \cdot R_2 + R_2 \cdot R_3 + R_1 \cdot R_3}{R_1} \\
    R_b &= \frac{R_1 \cdot R_2 + R_2 \cdot R_3 + R_1 \cdot R_3}{R_2} \\
    R_c &= \frac{R_1 \cdot R_2 + R_2 \cdot R_3 + R_1 \cdot R_3}{R_3}
\end{align}

\section{Potentiometer}
Ein Potentiometer ist ein verstellbarer Widerstand, der durch drehen oder schieben angepasst werden kann. \\
Es ist, vereinfacht gesagt, eine Serienschaltung zweier Widerstände:
\begin{center}
\begin{circuitikz}
    \draw (0,0) to[R=$R_1$] ++(2,0);
    \draw (2,0) to[R=$R_2$] ++(2,0);
    \draw (0,2) to[pR=$P_1$] ++ (4,0);
\end{circuitikz}
\end{center}