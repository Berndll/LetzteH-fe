\documentclass[paper=a4, 12pt]{scrreprt}
%% Encoding UTF8
\usepackage[utf8]{inputenc}
%%Use Source Sans Pro Textstyle
\usepackage[default]{sourcesanspro}
%% 8 Bit Aufloesung der Buchstaben
\usepackage[T1]{fontenc}
%% Seitenraender
%\usepackage[scale=0.72]{geometry}
\usepackage[scale=0.72, twoside, bindingoffset=2mm]{geometry}

\usepackage[onehalfspacing]{setspace}

%% Spracheinstellungen
% \usepackage[english, naustrian]{babel} % your native language must be the last one!!
\usepackage[naustrian, english]{babel} % your native language must be the last one!!
%% erweiterte Farbenpalette
\usepackage[dvipsnames]{xcolor}
%% Abbildungen
\usepackage{graphicx}
%%Tabelen mit Farbe (cellcolor)
\usepackage{tabulary}
\usepackage{colortbl}
\PassOptionsToPackage{dvipsnames,svgnames,table}{xcolor}
%% Tabellen (erweitert)
\usepackage{tabularx}
%% TikZ + Circuit-TikZ (fuer Schaltungen)
\usepackage[europeanresistors, europeaninductors]{circuitikz}
%% Nuetzliche TikZ Libraries
\usetikzlibrary{arrows, automata, positioning}
%% mathematik
\usepackage{amsmath, amssymb}
%%Formelbeschreibung
\newenvironment{conditions}
  {\par\vspace{\abovedisplayskip}\noindent\begin{tabular}{>{$}l<{$} @{${}...{}$} l}}
  {\end{tabular}\par\vspace{\belowdisplayskip}}
%\usepackage{mathtools}	
%% pdf-einbindung
\usepackage{pdfpages}
%% scource-code einbindung
\usepackage{listings, scrhack} %scrhack vermeidet Umschaltung auf KOMA
% Floats..
\usepackage{courier}
%% euro-symbol
\usepackage{eurosym}
%% landcsape-seiten ermöglichen
\usepackage{lscape}

%% Todos
\setlength{\marginparwidth}{2cm}
\usepackage[]{todonotes}

%% Ganttdiagramme
\usepackage{pgfgantt}

%% Subfigures
\usepackage[lofdepth]{subfig}

%% Für Quellenangaben unter Bildern
\newcommand*{\quelle}[1]{\par\raggedleft\footnotesize Quelle:~#1}

%% Listings (Code)
\usepackage{listings}
\usepackage{color}

%% CUSTOM PACKAGES
\usepackage{tikz-timing}
\usepackage{pgfplots}
\usepackage{longtable}
\usepackage{mdframed}
\usepackage{booktabs}
\usepackage{multirow}
\usepackage{biblatex}
\usepackage{siunitx}
\usepackage{graphicx,wrapfig,lipsum}
\usepackage{gensymb}

\title{Letzte Hüfe}
\author{xB Fucking HELS}
\date{\today}

\begin{document}
    \maketitle
    \pagebreak

    \tableofcontents
    \pagebreak

    \chapter{Grundkonzepte}

\section{Grundeinheiten}

\setlength{\tabcolsep}{4pt} % Default value: 6pt
\renewcommand{\arraystretch}{1.25} % Default value: 1
\begin{table}[!htb]
    \centering
    \begin{tabular}{|c|c|c|c|}
        \hline
        \textbf{SI-Einheiten}   & \textbf{Bedeutung}        & \textbf{Einheit}                  & \textbf{Zusammenhang}              \\ \hline
        U                       & Spannung                  & Volt (V)                          & -                                  \\
        I                       & Strom                     & Ampere (A)                        & -                                  \\
        R                       & Widerstand                & Ohm ($\Omega$)                    & -                                  \\
        G                       & Leitwert                  & Siemens (S)                       & $\frac{1}{R}$                      \\
        P                       & Leistung                  & Watt (W)                          & $U\cdot I$                         \\
        C                       & Kapazität                 & Farad (F)                         & $C\cdot s$                         \\
        Q                       & Ladung                    & Coulomb (C)                       & $C \cdot U$                        \\
        L                       & Induktivität              & Henry (H)                         & -                                  \\
        f                       & Frequenz                  & Hertz (Hz)                        & $s^{-1}$                           \\ 
        $\omega$                & Kreisfrequenz             & (rad/s)                           & -                                  \\
        W                       & Arbeit                    & Joule (J)                         & $N \cdot m$                        \\
        F                       & Kraft                     & Newton (N)                        & $V \cdot A \cdot s \cdot m^{-1}$   \\ 
        p                       & Druck                     & Pascal (Pa)                       & $N \cdot m^2$                      \\
        $\varphi$               & Potenzial                 & Volt (V)                          & $\frac{W}{A}$                      \\
        \hline
        H                       & magnetische Feldstärke    & Strom pro Meter ($\frac{A}{m}$)   & -                                  \\ 
        E                       & Elektrische Feldstärke    & $\frac{V}{m}$                     & $\frac{F}{Q}$                      \\
        $\Psi$                  & Elektrischer Fluss        & Coulomb $C$                       & -                                  \\
        $\phi$                  & magnetischer Fluss        & Weber (Wb)                        & $V \cdot s$                        \\
        D                       & Elektrische Flussdichte   & $\frac{C}{m^2}$                   & $\frac{\Psi}{A^2}$                 \\
        B                       & magnetische Flussdichte   & Tesla (T)                         & $C \cdot U$                        \\
        \hline
    \end{tabular}
    \caption{Grundeinheiten}
\end{table}

\newpage

\section{Konstanten}

\setlength{\tabcolsep}{4pt} % Default value: 6pt
\renewcommand{\arraystretch}{1.25} % Default value: 1
\begin{table}[!htb]
    \centering
    \begin{tabular}{|c|c|c|c|}
        \hline
        \textbf{Konstanten} & \textbf{Bedeutung}    & \textbf{Einheit} & \textbf{Wert teilweise gerundet} \\ \hline
        c               & Lichtgeschwindigkeit (Vakuum)   & $\frac{m}{s}$  & 299 792 458    \\
        e               & Elementarladung    & C  & $1,602 \cdot 10^{-19}$    \\
        $\mu \textsubscript{0}$               & Magnetische Feldkonstante   & $\frac{H}{m}$  & $4\cdot \pi \cdot 10^{-7}$  \\
        $\epsilon \textsubscript{0}$    & Permittivität   & $\frac{F}{m}$  & $8,854 \cdot 10^{-12}$  \\
        Cu    & Leitfähigkeit Kupfer    & $\frac{S \cdot m}{mm^{2}}$  & 56  \\
        \hline

    \end{tabular}
    \caption{Konstanten}
    % \label{Grundkonzepte/Einheiten/Tabelle}
\end{table}

\section{Präfixe}

\setlength{\tabcolsep}{4pt} % Default value: 6pt
\renewcommand{\arraystretch}{1.25} % Default value: 1
\begin{table}[!htb]
    \centering
    \begin{tabular}{|c|c|c|c|}
        \hline
        \textbf{Vorsatz} & \textbf{Vorsatzzeichen}        & \textbf{Faktor} & \textbf{Wert} \\ \hline
        Exa                & E                    & $10^{18}$           & 1 000 000 000 000 000 000  \\
        Peta               & P                    & $10^{15}$           & 1 000 000 000 000 000      \\
        Tera               & T                    & $10^{12}$           & 1 000 000 000 000          \\
        Giga               & G                    & $10^{9}$            & 1 000 000 000              \\
        Mega               & M (meg)              & $10^{6}$            & 1 000 000                  \\
        Kilo               & k                    & $10^{3}$            & 1 000                      \\
        Hekto              & h                    & $10^{2}$            & 100                        \\
        Deka               & da                   & $10^{1}$            & 10                         \\
        Dezi               & d                    & $10^{-1}$           & 0,1                        \\ 
        Zenti              & c                    & $10^{-2}$           & 0,01                       \\
        Milli              & m                    & $10^{-3}$           & 0,001                      \\
        Mikro              & $\mu$                & $10^{-6}$           & 0,000 001                  \\ 
        Nano               & n                    & $10^{-9}$           & 0,000 000 001              \\
        Piko               & p                    & $10^{-12}$          & 0,000 000 000 001          \\
        Femto              & f                    & $10^{-15}$          & 0,000 000 000 000 001      \\
        \hline

        \hline
    \end{tabular}
    \caption{Präfixe}
    % \label{Grundkonzepte/Einheiten/Tabelle}
\end{table}

    \chapter{Widerstand}

\section{Ohm'sches Gesetz}
Der Zusammenhang zwischen Spannung $U$, Strom $I$ und Widerstand $R$:
\begin{align}
    U = R\cdot I \hspace{1cm} I = \frac{U}{R} \hspace{1cm} R = \frac{U}{I}
\end{align}

Der Leitwert $G$ wird in Siemens ($S$) angegeben und ist der Kehrwert des Widerstands:
\begin{align}
    G = \frac{1}{R} = \frac{I}{U}
\end{align}

\section{Serienschaltung}
\begin{center}
\begin{circuitikz}
    \draw (0,0) to[R=$R_1$] ++(2,0);
    \draw (2,0) to[R=$R_2$] ++(2,0);
\end{circuitikz}
\end{center}

\begin{align}
    R_g &= \sum_{i=0}^{N} R_i = R_1 + R_2 + ...
\end{align}

\section{Parallelschaltung}
\begin{center}
\begin{circuitikz}
    \draw (0,0) to[R=$R_1$] ++(2,0);
    \draw (0,-2) to[R=$R_2$] ++(2,0);

    \draw[black] (0,0) -- (0,-2);
    \draw[black] (2,0) -- (2,-2);

    \draw[black] (-1,-1) -- (0,-1);
    \draw[black] (2,-1) -- (3,-1);
\end{circuitikz}
\end{center}

\begin{align}
    \frac{1}{R_g} &= \sum_{i=0}^{N} \frac{1}{R_i} = \frac{1}{R_1} + \frac{1}{R_2} + ...
\end{align}

Bei zwei parallel geschaltenen Widerständen gilt auch:
\begin{align}
    R_g = \frac{R_1 \cdot R_2}{R_1 + R_2}
\end{align}

\section{Spannungsteiler}
Die Spannung wird auf zwei Widerstände (bzw. Lasten) aufgeteilt. Über die Kirchhoff'schen Gesetze der Knoten- und Maschenregel können dadurch die einzelnen Spannungen, die auf den Widerständen abfallen berechnet werden.

\subsubsection*{Beispiel}
\begin{center}
\begin{circuitikz}
    \draw (0,0)
        to[R=$R_1$] ++(2,0)
        to[R=$R_2$] ++(2,0);
\end{circuitikz}
\end{center}

Es gilt:
\begin{align}
    U_{R_1} &= U_g \cdot \frac{R_1}{R_1 + R_2} \\
    U_{R_2} &= U_g \cdot \frac{R_2}{R_1 + R_2} \\
\end{align}

Da die Widerstände in Serie geschalten wurden, sind die Einzelströme gleich groß, d.h.:
\begin{align}
    I_{R_1} = I_{R_2}
\end{align}

\section{Leitungswiderstand}
\begin{align}
    R &= \frac{\rho \cdot l}{A} \\
    G &= \frac{A}{\rho \cdot l}
\end{align}
\begin{itemize}
    \item $\rho$ ... materialspezifischer Widerststand in [$\frac{\Omega \cdot mm^2}{m}$] oder [$\frac{mm^2}{S \cdot m}$]
    \item $l$ ... Länge der Leitung in [$m$]
    \item $A$ ... Querschnittsfläche der Leitung in [$m^2$]
\end{itemize}

\section{Sterndreiecktransformation}
Die Sterndreiecktransformation kann verwendet werden, um das Arbeiten mit gewissen Widerstandsnetzwerken zu erleichtern.

% \subsubsection*{Stern}
\begin{center}
\begin{tikzpicture}
    \begin{scope}[xshift=-3cm]    
        \draw (0,0) to[R=$R_2$] ++(330:2);
        \draw (0,0) to[R=$R_1$] ++(210:2);
        \draw (0,0) to[R=$R_3$] ++ (90:2);
    \end{scope}

    \begin{scope}[
            xshift=3cm,
            yshift=-1.25cm
        ]
        \draw  (1.5,0) to[R=$R_a$] ++(120:3.5);
        \draw (-2,0) to[R=$R_b$] ++ (60:3.5);
        \draw (-2,0) to[R=$R_c$] ++  (0:3.5);
    \end{scope}
\end{tikzpicture}
\end{center}

\subsubsection*{Stern-zu-Dreieck}
\begin{align}
    R_1 &= \frac{R_b \cdot R_c}{R_a + R_b + R_c} \\
    R_2 &= \frac{R_a \cdot R_c}{R_a + R_b + R_c} \\
    R_3 &= \frac{R_a \cdot R_b}{R_a + R_b + R_c}
\end{align}

\subsubsection*{Dreieck-zu-Stern}
\begin{align}
    R_a &= \frac{R_1 \cdot R_2 + R_2 \cdot R_3 + R_1 \cdot R_3}{R_1} \\
    R_b &= \frac{R_1 \cdot R_2 + R_2 \cdot R_3 + R_1 \cdot R_3}{R_2} \\
    R_c &= \frac{R_1 \cdot R_2 + R_2 \cdot R_3 + R_1 \cdot R_3}{R_3}
\end{align}

\section{Potentiometer}
Ein Potentiometer ist ein verstellbarer Widerstand, der durch drehen oder schieben angepasst werden kann. \\
Es ist, vereinfacht gesagt, eine Serienschaltung zweier Widerstände:
\begin{center}
\begin{circuitikz}
    \draw (0,0) to[R=$R_1$] ++(2,0);
    \draw (2,0) to[R=$R_2$] ++(2,0);
    \draw (0,2) to[pR=$P_1$] ++ (4,0);
\end{circuitikz}
\end{center}

    \chapter{Kirchhoff}
Kirchhoff hat zwei fundamentale Regeln bzw. Gesetze aufgestellt: \\
Die \textbf{Knotenregel} und die \textbf{Maschenregel}.

\section{Knotenregel}
Die Summe aller Ströme bei einem Knotenpunkt ist $0$, d.h. Ströme die hineinfließen, müssen auch hinausfließen:
\begin{align}
    \sum_{i = 0}^{N} I_i = 0
\end{align}

Diese Regel besagt, dass alle Ströme in einer Serienschaltung die gleichen sein müssen. Ebenso müssen Ströme die zu einem zusammenlaufen, aufsummiert werden, um den Ausgangsstrom zu ermitteln.

\section{Maschenregel}
Die Summer aller Spannungen in einer Masche ist $0$.

\begin{align}
    \sum_{i = 0}^{N} U_i = 0
\end{align}

Laut dieser Regel sind Spannungen einer Parallelschaltung immer gleich groß. Auch gilt, dass die Summe von Spannungen einer Serienschaltungen aufsummiert werden müssen, um die Gesamtspannung zu berechnen.

    \chapter{Leistung}
\begin{itemize}
    \item \underline{Wirkleistung $P$} \\
    Tatsächlich umgesetzte Energie in Watt [W]
    \item \underline{Blindleistung $Q$} \\
    Unerwünschte bzw. nicht nutzbare Energie in Volt-Ampere Relativ [var]
    \item \underline{Scheinleistung $S$} \\
    Gesamtleistung in Volt-Ampere [VA]
\end{itemize}

\section{Leistung bei Gleichstrom}
Allgemein gilt:

\begin{align}
    P&=U\cdot I \\
    P&=\frac{U^2}{R} \\
    P&=I^2\cdot R    
\end{align}

\begin{itemize}
    \item $P$ ... Leistung
    \item $U$ ... Spannung
    \item $I$ ... Strom
    \item $R$ ... Widerstand
\end{itemize}

\newpage

\subsection{Wirkleistung $P$}
\begin{align}
    P = U_w \cdot I = U \cdot I \cdot cos(\varphi)
\end{align}
\begin{itemize}
    \item $P$ ... Wirkleistung in [$W$]
    \item $U_w$ ... Wirkkomponente der Spannung in [$V$]
\end{itemize}

\subsection{Blindleistung $Q$}
\begin{align}
    Q = U_b \cdot I = U \cdot I \cdot sin(\varphi)
\end{align}
\begin{itemize}
    \item $Q$ ... Blindleistung in [$var$]
    \item $U_b$ ... Blindkomponente der Spannung in [$V$]
\end{itemize}
Die \textbf{induktive} Blindleistung ist \textbf{positiv}: $sin(\varphi) > 0 \rightarrow Q_L > 0$ \\
Die \textbf{kapazitive} Blindleistung ist \textbf{negativ}: $sin(\varphi) < 0 \rightarrow Q_C < 0$ \\

\subsection{Scheinleistung $S$}
\begin{align}
    S = U \cdot I
\end{align}
\begin{itemize}
    \item $S$ ... Scheinleistung in [$VA$]
\end{itemize}

\newpage

\section{Leistung bei Wechselstrom}
Bei einem sinusförmigen Verlauf der Spannung $u$ und des Stroms $i$ gelten folgende Gleichungen:
\begin{align}
    u = \hat{U} \cdot sin(\omega \cdot t) \hspace{1cm} i = \hat{I} \cdot sin(\omega \cdot t - \varphi)
\end{align}
Werden die Momentanwerte von $u$ und $i$ miteinander multipliziert, erhält man den Momentanwert der Leistung $p$:
\begin{align}
    p = u \cdot i = \hat{U} \cdot sin(\omega \cdot t) \cdot \hat{I} \cdot sin(\omega \cdot t - \varphi)
\end{align}
oder:
\begin{align}
    p = U_{eff} \cdot I_{eff} \cdot cos(\varphi) - U_{eff} \cdot I_{eff} \cdot (2\cdot \omega \cdot t - \varphi)
\end{align}

\begin{itemize}
    \item $p$ ... Momentanwert der Wechselstromleistung in [$W$]
    \item $U$ ... Effektivwert der Spannung in [$V$]
    \item $I$ ... Effektivwert des Stromes in [$A$]
\end{itemize}

\subsection{Leistungsfaktor}
Der Leistungsfaktor $cos(\varphi)$ gibt an, welchen Anteil die Wirkleistung an der Scheinleistung hat. Er erreicht bei ohmschen Lasten den maximalen Wert $1$:
\begin{align}
    P &= U \cdot I \cdot cos(\varphi) = S \cdot cos(\varphi) \\
    cos(\varphi) &= \frac{P}{S} 
\end{align}
\begin{itemize}
    \item $P$ ... Wirkleistung in [$W$]
    \item $S$ ... Scheinleistung in [$VA$]
    \item $cos(\varphi)$ ... Leistungsfaktor
\end{itemize}

\subsection{Kompensation}
\textit{GdE2 S.102} \\
Ein ohmsch-induktiver Verbraucher, wie z.B. ein Elektromotor, entnimmt dem Netz nicht nur Wirkleistung, sondern zum Aufbau des Magnetfeldes auch induktive Blindleistung. Der fließende Blindstromanteil belastet das Netz mit unerwünschten Spannungsabfällen und erhöhten Übertragungsverlusten. Ein parallel zum Verbraucher liegender Kondensator kompensiert die aus dem Netz bezogene Blindleistung und verbessert den Leistungsfaktor. Da die zugeführte Wirkleistung unverändert bleibt, ergibt sich folgendes Leistungsdreieck:

\begin{center}
\begin{tikzpicture}[>=latex, scale = 2]
    \draw[style=help lines] (0,0) (3,2);

    \coordinate (pointer0) at (45:3);

    \coordinate (posX) at (0:4);
    \coordinate (posY) at (90:3);
    \coordinate (negX) at (180:1);
    \coordinate (negY) at (270:1);

    % \draw[->,thick,Green] (0,0) -- (pointer0) node[right, xshift=3pt, Green] {$S$};
    \draw[->,thick,red] (0,0.05) -- (3,0.05) node[below,  red] {$P_{ZU}$};
    
    \draw[->,thick,cyan] (3.05,0) -- (3.05,1) node[right, midway, cyan] {$Q$};
    \draw[->,thick,cyan] (3.05,2) -- (3.05,1) node[right, midway, cyan] {$Q_c$};
    \draw[->,thick,blue] (3,0) -- (3,2) node[left, midway,yshift = 7pt, blue] {$Q_M$};

    \draw[->,thick,green!50!black] (0,0) -- (3,2) node[above, green!50!black] {$S$};
    \draw[->,thick,green!50!black] (0,0) -- (3,1) node[above,midway,yshift = 5pt,xshift = 15pt, green!50!black] {$S`$};

    \draw[ultra thick, ->,green!50!black] (1,0) arc (0:45:0.7)node[right, midway,xshift = 3pt,yshift = -2pt ,green!50!black]{$\varphi$};
    \draw[ultra thick, ->,green!50!black] (1.7,0) arc (0:45:0.75)node[right, midway,xshift = 3pt,yshift = 0pt ,green!50!black]{$\varphi`$};
    

    \draw[->,thick,black] (0,0) -- (posX) node[right,xshift=3pt] {$P$};
    \draw[->,thick,black] (0,0) -- (posY) node[right,xshift=3pt] {$Q$};
    \draw[->,thick,black] (0,0) -- (negX);
    \draw[->,thick,black] (0,0) -- (negY);
\end{tikzpicture}
\end{center}

Wird der Leistungsfaktor von $cos(\varphi)$ auf $cos(\varphi')$ verbessert, folgt aus dem Leistungsdreieck mit der Blindleistung $Q_M$ des Verbrauchers und der Blindleistung $Q$ des Netzes die erforderliche Blindleistung $Q_C$:
\begin{align}
    Q_C = Q - Q_M
    \hspace{1cm}
    Q = P_{ZU} \cdot tan(\varphi')
    \hspace{1cm}
    Q_M = P_{ZU} \cdot tan(\varphi) \\
    \Rightarrow Q_C = P_{ZU} \cdot (tan(\varphi') - tan(\varphi))
\end{align}

\begin{itemize}
    \item $Q_C$ ... Blindleistung des Kondensators in [$var$]
    \item $P_{ZU}$ ... Wirkleistung des Verbrauchers in [$W$]
    \item $\varphi$ ... Phasenwinkel ohne Kompensation in [$\degree$]
    \item $\varphi'$ ... Phasenwinkel mit Kompensation in [$\degree$]
\end{itemize}

\begin{center}
\begin{tikzpicture}[>=latex, scale = 2]
    \draw[style=help lines] (0,0) (3,2);

    \coordinate (pointer0) at (45:3);

    \coordinate (posX) at (0:4.5);
    \coordinate (posY) at (90:1);
    \coordinate (negX) at (180:1);
    \coordinate (negY) at (270:3);


    \draw[->,thick,black] (0,0) -- (posX) node[right,xshift=3pt] {$+$};
    \draw[->,thick,black] (0,0) -- (posY) node[right,xshift=3pt] {$+j$};
    \draw[->,thick,black] (0,0) -- (negX);
    \draw[->,thick,black] (0,0) -- (negY);

    \draw[->,thick,red] (0,0.05) -- (3,0.05) node[right,yshift=7pt, red] {$I_w$};
    \draw[->,thick,red] (0,0) -- (3,-1) node[right, red] {$I`$};
    \draw[->,thick,red] (0,0) -- (3,-2.5) node[right,yshift = 7pt, red] {$I$};
    \draw[thick,red, dashed] (3,0.05) -- (3,-2.5) ;
    \draw[thick,cyan, ->] (3,0.05) -- (4,0.05) node[right,yshift = 7pt, cyan] {$U$};

    \draw[ultra thick, ->,green!50!black] (1,-0.8) arc (-45:45:0.6) node[right, midway,xshift = 3pt,yshift = -17pt ,green!50!black]{$\varphi$};
    \draw[ultra thick, ->,green!50!black] (1.8,-0.6) arc (-45:45:0.45) node[right, midway,xshift = 3pt,yshift = 0pt ,green!50!black]{$\varphi`$};
    
\end{tikzpicture}
\end{center}
    
\begin{center}
\begin{tikzpicture}[>=latex, scale = 2]
    \draw[style=help lines] (0,0) (3,2);

    \coordinate (pointer0) at (45:3);

    \coordinate (posX) at (0:4.5);
    \coordinate (posY) at (90:1.5);
    \coordinate (negX) at (180:1);
    \coordinate (negY) at (270:3);


    \draw[->,thick,black] (0,0) -- (posX) node[right,xshift=3pt] {$+$};
    \draw[->,thick,black] (0,0) -- (posY) node[right,xshift=3pt] {$+j$};
    \draw[->,thick,black] (0,0) -- (negX);
    \draw[->,thick,black] (0,0) -- (negY);

    \draw[->,thick,red] (0,0) -- (3,1) node[right, red] {$I`$};
    \draw[->,thick,red] (0,0.05) -- (3,0.05) node[right,yshift=7pt, red] {$I_w$};
    \draw[->,thick,red] (0,0) -- (3,-2.5) node[right,yshift = 7pt, red] {$I$};
    \draw[thick,red, dashed] (3,1) -- (3,-2.5) ;
    \draw[thick,cyan, ->] (3,0.05) -- (4,0.05) node[right,yshift = 7pt, cyan] {$U$};

    \draw[ultra thick, ->,green!50!black] (1,-0.8) arc (-45:45:0.6) node[right, midway,xshift = 3pt,yshift = -17pt ,green!50!black]{$\varphi$};
    \draw[ultra thick, ->,green!50!black] (1.8,0.6) arc (45:-45:0.4) node[right, midway,xshift = 3pt,yshift = 0pt ,green!50!black]{$\varphi`$};
    
\end{tikzpicture}
\end{center}

    \chapter{Quellen}

\section{Spannungsquelle}

\section{Stromquelle}

\section{Überlagerungsprinzip}

\section{Ersatzschaltbild}

    \chapter{Felder}

\section{Elektrisches Feld}

\section{Elektrischer Fluss}

\section{Magnetisches Feld}

\section{Magnetischer Fluss}

    \chapter{dB-Rechnung}
Die Darstellung in Dezibel (dB) findet man in der Elektronik beispielsweise bei \textbf{Bodediagrammen} und vor allem in der \textbf{Hochfrequenztechnik} Verwendung. 
Bei der Umrechnung und Darstellung in dB muss darauf geachtet werden, dass in \textbf{Leistungs-} und \textbf{Spannungsgrößen} unterteilt wird. Zu den sogenannten Leistungsgrößen gehören die Watt bzw. die Milliwatt. Ein häufiges Beispiel für Spannungsgrößen ist die Darstellung von Volt.\\

\textbf{Spannungsgrößen} werden mit dem \textbf{Faktor 20} multipliziert; \textbf{Leistungsgrößen} lediglich mit dem \textbf{Faktor 10}.

\section{Rechenregeln}
Im Generellen werden alle Rechenoperatoren um eine Stufe herabgesetzt:

\begin{itemize}
    \item \textbf{Faltung}($*$) $\rightarrow$ \textbf{Multiplikation}($\cdot$)
    \item \textbf{Multiplikation}($\cdot$) $\rightarrow$ \textbf{Addition}($+$)
    \item \textbf{Addition}($+$) $\rightarrow$ undefiniert
\end{itemize}

\underline{Beispiele}
\begin{itemize}
    \item $2\cdot 1000\Rightarrow 3[dBW]+30[dBW]$
    \item $f_1 * f_2 \Rightarrow f_1[dB]\cdot f_2[dB]$
\end{itemize}

\section{Besonderheiten bei Leistungsgrößen}
Bei Leistungsgrößen können je nach Aufgabengebiet die sogenannten \textbf{dB-Watt (dBW)} oder die \textbf{dB-Milliwatt (dBm)} benötigt werden.\\
Diese dBm werden hauptsächlich in der Hochfrequenztechnik eingesetzt, um kleine Leistungen angemessen darstellen zu können.\\

Dadurch, dass bei dBm mit dem Faktor 1.000 multipliziert wird, liegt \textbf{1dBm} um \textbf{30dB unter 1dBW}.\\ 

\underline{Beispiele}
\begin{itemize}
    \item $3[dBW] = 33[dBm]$
    \item $0[dBm] = -30[dBW]$
\end{itemize}

\section{Allgemeine Formeln}
Hier sind allgemeine Formeln für Spannungs- und Leistungsgrößen angegeben. Der Folgepfeil zeigt eine \underline{vereinfachte} Form.

\subsection{Spannungsgrößen}
\begin{align}
    U[dBV]=20\cdot log_{10}(\frac{U[V]}{U_0[V]}) &\Rightarrow U[dBV]=20\cdot log_{10} U[V] \\
    U[V]=10^{\frac{U[dBV]}{20}}\cdot U_0[V] &\Rightarrow 10^{\frac{U[dBV]}{20}}
\end{align}

\subsection{Leistungssgrößen}
\begin{align}
    P[dBW]=10\cdot log_{10}(\frac{P[W]}{1[W]}) &\Rightarrow P[dBW]=10\cdot log_{10} P[W] \\
    P[W]=10^{\frac{P[dBW]}{10}}\cdot 1[W] &\Rightarrow 10^{\frac{U[dBV]}{10}}
\end{align}

\section{Spezielle Formeln}
\begin{align}
    U[dBV]=20\cdot log_{10}(\frac{U_a[V]}{U_e[V]})
\end{align}
Oben ist die Formel zum Umrechnen der \textbf{Übertragungsfunktion} für das \textbf{Bodediagramm} in dB angegeben. Da es sich bei der Übertragungsfunktion um \textbf{Spannungsgrößen} handelt wird mit dem \textbf{Faktor 20} multipliziert.

\begin{align}
    P[dBm]&=10\cdot log_{10}(\frac{P[W]}{1[mW]}) \\
    P[W]&=1[mW]\cdot 10^{\frac{P[dBm]}{10}}
\end{align}
Oben ist die Formel zum Umrechnen von \textbf{Watt} und \textbf{dBm} (siehe Unterkapitel „Besonderheiten bei Leistungsgrößen“).

\section{Häufige Zahlenwerte}

\begin{table}[!htb]
    \centering
    \begin{tabular}{|l|c|c|}
        \hline
        \textbf{Normalraum} & \textbf{dBW}  & \textbf{dBV}  \\ \hline
        $0,001$             & $-30$         & $-60$         \\
        $0,01$              & $-20$         & $-40$         \\
        $0,1$               & $-10$         & $-20$         \\
        $0,5$               & $\approx-3$   & $\approx-6$   \\
        $1$                 & $0$           & $0$           \\
        $2$                 & $\approx 3$   & $\approx 6$   \\
        $10$                & $10$          & $20$          \\
        $100$               & $20$          & $40$          \\
        $1000$              & $30$          & $60$          \\ \hline
    \end{tabular}
\end{table}

    \chapter{Wechselstromtechnik}

\section{Komplexe Zahlen}
Komplexe Zahlen sind die Erweiterung der Realen Zahlen $\mathbb{R}$:
\begin{align}
    \mathbb{N}\rightarrow\mathbb{Z}\rightarrow\mathbb{R} (\mathbb{Q}+\mathbb{I})\rightarrow\mathbb{C}
\end{align}

Die Defintion $j=\sqrt{-1}$ ist hierbei besonders wichtig.

\vspace{0.5cm}

\textbf{Beispiel}
\begin{align}
    &x^2=-9         \\
    &x^2=j^2\cdot 9\hspace{0.25cm}|\sqrt{}\\
    &\Rightarrow x_1=3j\hspace{0.5cm} x_2=-3j
\end{align}

\newpage

Eine komplexe Zahl $\underline{z}$ besteht aus einem Realteil $a$ und einem Imaginärteil $b$

\begin{center}
\begin{tikzpicture}[>=latex]
    \draw[style=help lines] (0,0) (3,2);

    \coordinate (pointer) at (30:5); 
    % \coordinate (vec2) at (30:2.5);
    \coordinate (posX) at (0:5);
    \coordinate (posY) at (90:3);
    \coordinate (negX) at (180:1);
    \coordinate (negY) at (270:1);

    \draw[->,thick,black] (0,0) -- (pointer) node[right, xshift=3pt] {$\underline{z}$};
    % \draw at (15:2.5) node[left,top,xshift=3pt,yshift=3pt] {$\underline{|z|}$};
    \draw[->,thick,black] (0,0) -- (posX) node[right,xshift=3pt] {$Re(\underline{z})$};
    \draw[->,thick,black] (0,0) -- (posY) node[right,xshift=3pt] {$Im(\underline{z})$};
    \draw[->,thick,black] (0,0) -- (negX);
    \draw[->,thick,black] (0,0) -- (negY);

    % Re
    \draw[dashed,red] (pointer) -- +(-{sin(60)*5},0);
    \draw[dashed,red] (pointer) -- +(-{sin(60)*5}/2,0) node[above] {$a$};

    % Im
    \draw[dashed,blue] (pointer) -- +(0,-{sin(30)*5});
    \draw[dashed,blue] (pointer) -- +(0,-{sin(30)*5}/2) node[right] {$b$};

    \draw[->] (2,0) arc (0:30:2) node[midway,left,xshift=-3pt,yshift=-2pt] {$\varphi$};
\end{tikzpicture}
\end{center}

Der Betrag des Zeigers ($|\underline{z}|$) ist die Länge, $\varphi$ der Winkel zwischen der x-Achse und dem Zeiger.
\begin{align}
    &\underline{z}=a+jb=Re(\underline{z})+j Im(\underline{z})   \\
    &\underline{z}=|\underline{z}|\cdot e^{j\varphi}
\end{align}
Die Länge kann über den Pythagoras berechnet werden und der Winkel mit dem Arkustangens:
\begin{align}
    Re(\underline{z})&=|\underline{z}|\cdot cos(\varphi)                \\
    Im(\underline{z})&=|\underline{z}|\cdot sin(\varphi)                \\
    \varphi&=arctan(\frac{Im(\underline{z})}{Re(\underline{z})})        \\
    |\underline{z}|&=\sqrt{Re(\underline{z})^{2}+Im(\underline{z})^{2}}
\end{align}

\newpage

\subsection{Addition \& Subtraktion}
Die Summe \& Differenz kompler Zahlen $\underline{z_1}=a+jb$ und $\underline{z_2}=c+jd$ ist definiert als
\begin{align}
    \underline{z_1} + \underline{z_2} = (a+c) + j(c+d) \hspace{1cm} \underline{z_1} - \underline{z_2} = (a+c) - j(c+d)
\end{align}

Es werden Real- und Imaginärteile addiert bzw. subtrahiert.\\

Grafisch können Zahlen in Zeigerdarstellung wie Vektoren addiert bzw. subtrahiert werden. D.h. beim Addieren wird das Ende eines Zeigers wird an die Spitze des anderen gehängt.\\

\underline{Beispiel} \\
\todo{Radiant zu Grad konvertieren?}
$\underline{x}=3\cdot e^{j\frac{\pi}{3}}$ \\
$\underline{y}=3\cdot e^{j\frac{8\pi}{12}}$ \\

Gesucht: $\underline{z}$\\
$\underline{z}=\underline{x}+\underline{y}$

\begin{center}
\begin{tikzpicture}[>=latex]
    \draw[style=help lines] (0,0) (3,2);

    \coordinate (pointer0) at (30:3);
    \coordinate (pointer1) at (240:3);
    % \coordinate (vec2) at (30:2.5);
    \coordinate (posX) at (0:3);
    \coordinate (posY) at (90:2);
    \coordinate (negX) at (180:1);
    \coordinate (negY) at (270:2);

    \draw[->,thick,red] (0,0) -- (pointer0) node[right, xshift=3pt, red] {$\underline{x}$};
    \draw[->,thick,blue] (pointer0) -- +(pointer1) node[right, xshift=3pt, blue] {$\underline{y}$};
    \draw[->,thick,teal] (0,0) -- ($(pointer0) + (pointer1)$) node[below left, xshift=3pt, teal] {$\underline{z}$};

    \draw[->,thick,black] (0,0) -- (posX) node[right,xshift=3pt] {$Re(\underline{z})$};
    \draw[->,thick,black] (0,0) -- (posY) node[right,xshift=3pt] {$Im(\underline{z})$};
    \draw[->,thick,black] (0,0) -- (negX);
    \draw[->,thick,black] (0,0) -- (negY);
\end{tikzpicture}
\end{center}

\newpage

\subsection{Multiplikation}
Die Längen der Zeiger multiplizieren und die Winkel addieren:
\begin{align}
    \underline{y}\cdot\underline{z}=|\underline{y}|\cdot|\underline{z}|\cdot e^{j\cdot(\varphi_{\underline{y}}+\varphi_{\underline{z}})}
\end{align}

\subsection{Division}
Die Längen der Zeiger dividieren und die Winkel subtrahieren:
\begin{align}
    \frac{\underline{y}}{\underline{z}}=\frac{|\underline{y}|}{|\underline{z}|}\cdot e^{j\cdot(\varphi_{\underline{y}}-\varphi_{\underline{z}})}
\end{align}

\subsection{Konjugiert Komplexe Zahlen}
Konjugiert-Komplexe Zahlen sind besonders wichtig, wenn man mit komplexen Zahlen rechnen möchte. Um die konjugiert-komplexe Zahl zu ermitteln, wird nur das Vorzeichen des Imaginärteils der Zahl umgedreht; sie wird als $\underline{z}^*$ angeschrieben.
\begin{align}
    \underline{z}&=a+jb=|\underline{z}|\cdot e^{j\varphi}       \\
    \underline{z}^*&=a-jb=|\underline{z}|\cdot e^{-j\varphi}
\end{align}

Visuell ist es das Gleiche, als wenn man den Punkt auf der x-Achse spiegelt.

\begin{center}
\begin{tikzpicture}[>=latex]
    \draw[style=help lines] (0,0) (3,2);

    \coordinate (pointer) at (30:3);
    \coordinate (negPointer) at (330:3);
    % \coordinate (vec2) at (30:2.5);
    \coordinate (posX) at (0:5);
    \coordinate (posY) at (90:2);
    \coordinate (negX) at (180:1);
    \coordinate (negY) at (270:2);

    \draw[->,thick,black] (0,0) -- (pointer) node[right, xshift=3pt] {$\underline{z}$};
    \draw[->,thick,black] (0,0) -- (negPointer) node[right, xshift=3pt] {$\underline{z}^*$};

    \draw[->,thick,black] (0,0) -- (posX) node[right,xshift=3pt] {$Re(\underline{z})$};
    \draw[->,thick,black] (0,0) -- (posY) node[right,xshift=3pt] {$Im(\underline{z})$};
    \draw[->,thick,black] (0,0) -- (negX);
    \draw[->,thick,black] (0,0) -- (negY);
\end{tikzpicture}
\end{center}

\newpage

\section{Zeigerdiagramm}
Mit Zeigerdiagrammen kann man sinusförmige Funktion übersichtlicher darstellen. Die Zeiger folgen dem Einheitskreis und sollen zeigen, wie sich die Funktion zeitlich verhält.

\begin{tikzpicture}[>=latex]
    \draw[style=help lines] (0,0) (3,2);

    % Plot
    \coordinate (posX) at (0:11);
    \coordinate (posY) at (90:3);
    \coordinate (negX) at (180:3);
    \coordinate (negY) at (270:3);

    \draw[->,thick,black] (0,0) -- (posX) node[right,xshift=3pt] {$t$};
    \draw[->,thick,black] (0,0) -- (posY) node[right,xshift=3pt] {$y(t)$};
    \draw[->,thick,black] (0,0) -- (negX);
    \draw[->,thick,black] (0,0) -- (negY);

    \draw[dashed] (90:2) -- +(11,0);
    \draw[dashed] (270:2) -- +(11,0);
    \draw[->,thick,black] (0:3)  -- +(0,3);
    \draw[->,thick,black] (0:3)  -- +(0,-3);

    % Pointer
    \draw[->,thick,blue] (0,0.05) -- +(2,0) node[above right] {$t_0$};
    \draw[->,thick,blue] (-0.05,0) -- +(0,2) node[above left] {$t_1$};
    \draw[->,thick,blue] (0,0) -- (135:2) node[left] {$t_2$};
    \draw[->,thick,blue] (0,0) -- (270:2) node[below left] {$t_3$};

    \draw[->,thick,red] (0.05,0) -- +(0,2) node[above right] {$t_0$};
    \draw[->,thick,red] (0,0) -- (180:2) node[above left] {$t_1$};
    \draw[->,thick,red] (0,0) -- (225:2) node[left] {$t_2$};
    \draw[->,thick,red] (0,-0.05) -- +(2,0) node[below left] {$t_3$};

    % Circle
    \draw[-] (2,0) arc (0:360:2);
    \draw[->, thick] (2,0) arc (0:30:2) node[right] {$\omega$};

    % Sinewave
    \draw[thick, blue] (3,0) sin (5,2) cos (7,0) sin (9,-2) cos (11,0);
    % Cosinewave
    \draw[thick, red] (3,2) cos (5,0) sin (7,-2) cos (9,0) sin (11,2);

    % Points
    \filldraw[blue] (3,0)       circle (2pt) node[below left, blue] {$t_0$};
    \filldraw[blue] (5,2)       circle (2pt) node[above left, blue] {$t_1$};
    \filldraw[blue] (6,1.4142)  circle (2pt) node[below left, blue] {$t_2$};
    \filldraw[blue] (9,-2)      circle (2pt) node[below left, blue] {$t_3$};

    \filldraw[red] (3,2)       circle (2pt) node[below left, red]   {$t_0$};
    \filldraw[red] (5,0)       circle (2pt) node[below left, red]   {$t_1$};
    \filldraw[red] (8,-1.4142) circle (2pt) node[left, red]         {$t_2$};
    \filldraw[red] (9,0)       circle (2pt) node[above left, red]   {$t_3$};
\end{tikzpicture}

\newpage

\section{Impedanz}
Die Impedanz ist der "Widerstand" eines Systems, die aber auch die Frequenz einbezieht (weil der Imaginärteil nicht $0$ ist). Sie wird mit $z$ dargestellt.
\begin{align}
    &\underline{z}=\frac{\underline{U}}{\underline{I}}  \\
    &\underline{z}=R+jx
\end{align}

\textbf{Beispiel} \\
Kondensator: $C=1\mu F$\\
\begin{align}
    &\underline{z}_C=R+jx=\underline{y}_C                       \\
    &\underline{x}_C=\frac{1}{j\cdot\omega C}\cdot\frac{j}{j}
\end{align}
\begin{align}
    \underline{U}=5V\cdot e^{j\cdot 0} \hspace{1cm} f=1kHz
\end{align}

\section{Admittanz}
Die Admittanz ist der Kehrwert der Impedanz und sozusagen "der Leitwert, zum Widerstand". Sie wird mit dem Buchstaben $\underline{y}$ angeschrieben.
\begin{align}
    &\underline{y}=\frac{1}{\underline{z}}  \\
    &\underline{y}=G+jB
\end{align}

    \chapter{Lineare Bauteile}

\section{Kondensator}

\subsection{Kapazität}

\subsection{Ladung}

\section{Spule}

\subsection{Induktionsvorgänge}

\subsection{Kopplungsgrad}

\subsection{Induktivitäten}



\section{RLC Netzwerke}

\subsection{$\tau$-Messung}



\section{Resonanzkreise}

\subsection{Güte}

\subsection{Bandbreite}


\section{Übertragungsfunktion}
Die Übertragungsfunktion beschreibt das Ausgangs- im Vergleich zum Eingangssignal und ist definiert als
\begin{align}
    \underline{H}(j\omega)=\frac{\underline{U}_2}{\underline{U}_1}
\end{align}
wobei $\underline{U}_1$ der Eingang und $\underline{U}_2$ der Ausgang ist.

\subsection{Bodediagramm}
Das Bodediagramm zeigt das Verhalten eines Systems im logarithmischen Frequenzbereich. Es besteht aus Amplitudengang (in dB) und Phasengang (in °) und veranschaulicht die Übertragungsfunktion $H(j\omega)$.\\
Es könnte beispielsweise so aussehen:
\begin{tikzpicture}
    \begin{semilogxaxis}[
        xlabel={$\omega [rad]$},
        ylabel={$|\underline{H}(j\omega)|[dB]$},
        xmin=0.01, xmax=100,
        ymin=-40, ymax=10,
        xmode=log,
        grid=both,
        grid style={line width=.1pt, draw=gray!10},
        major grid style={line width=.2pt,draw=gray!50},
        minor tick num=5,
        width=10cm,
        height=8cm,
        ]
    
    \addplot[domain=0.01:100, blue, thick] {
        20 * log10(
            1/(x+1)
        )
    };
    \addlegendentry{Amplitude};
\end{semilogxaxis}
\end{tikzpicture}
    
\vspace{0.25cm}
    
\begin{tikzpicture}
\begin{semilogxaxis}[
        xlabel={$\omega [rad]$},
        ylabel={$\varphi [\degree]$},
        xmin=0.01, xmax=100,
        ymin=-90, ymax=0,
        xmode=log,
        grid=both,
        grid style={line width=.1pt, draw=gray!10},
        major grid style={line width=.2pt,draw=gray!50},
        minor tick num=5,
        width=10cm,
        height=8cm,
        ]
    
    \addplot[domain=0.01:100, red, thick] {-atan(x/10)};
    \addlegendentry{Frequenz};
    
    \end{semilogxaxis}
\end{tikzpicture}

    \chapter{Halbleiter}

\section{Dioden}
\subsection{Sperrkennlinie}
\subsection{Durchbruchsspannung}

\section{MOSFET}

\section{Bipolartransistor}

    \chapter{OPV-Schaltungen}

\section{Verstärker}

\subsection{Leerlaufverstärkung}
Merke, dass kein Strom in den OPV fließt; es gilt:
\begin{align}
    V = \frac{U_a}{U_d}
\end{align}
\begin{itemize}
    \item $V$ ist die Verstärkung
    \item $U_a$ ist die Ausgangsspannung
    \item $U_d$ ist die Differenzspannung (zwischen dem "+"- und "-"-Eingang).
\end{itemize}
\begin{center}
\begin{circuitikz}
    \coordinate (opAmp) at (0,0);
    \coordinate (opIn+) at ($(opAmp) + (-1.25,.5)$);
    \coordinate (opIn+Label) at ($(opIn+) + (-0.25,-0.1)$);
    \coordinate (opIn-) at ($(opAmp) + (-1.25,-.5)$);
    \coordinate (opIn-Label) at ($(opIn-) + (-0.25,0.1)$);
    \coordinate (opOut) at ($(opAmp) + (1.25,-0.1)$);

    \draw(opAmp) node[op amp, yscale=-1] {};
    \draw (1.25,-1.15) node[rground]{};


        
    \draw[->, thick, blue] (opIn+Label) -- (opIn-Label) node[above left, blue] {$U_D$};
    \draw[->, thick, blue] (opOut) -- +(0,-0.9) node[above right, blue] {$U_a$};

    \draw (-1,0.49) -- (-1.45,0.49);
    \draw (-1,-0.49) -- (-1.45,-0.49);

    \draw (1.25,0) circle (1.5pt);
	\draw (1.25,-1.1) circle (1.5pt); 
    \draw (-1.5,0.49) circle (1.5pt);
	\draw (-1.5,-0.49) circle (1.5pt); 
\end{circuitikz}
\end{center}
\subsection{Impedanzwandler}
Ist dazu da, um dem folgenden System mehr Strom liefern zu können.\\
Beachte, dass die Verstärkung hier $V=0$ ist weil $U_a=U_e$.
\begin{center}
\begin{circuitikz}
    \coordinate (opAmp) at (3,-0.5);
    \coordinate (opIn+) at ($(opAmp) + (-1.25,.5)$);
    \coordinate (opIn+Label) at ($(opIn+) + (-0,-0.1)$);
    \coordinate (opIn-) at ($(opAmp) + (-1.25,-.5)$);
    \coordinate (opIn-Label) at ($(opIn-) + (0,0.1)$);
    \coordinate (opOut) at ($(opAmp) + (1.25,0)$);

    \draw(opAmp) node[op amp, yscale=-1] {};
    \draw (0,-2.05) node[rground]{};
    \draw (5.5,-2.05) node[rground]{};

    \draw[black] (0.05,0) -- (opIn+);
    \draw[black] (opOut) -- +(1.2,0);
    \draw[black] (opOut) -- +(0,-1.5) -- +(-2.5,-1.5) -- (opIn-);

    \draw[->, thick, blue] (opIn+Label) -- (opIn-Label) node[above left, blue]    {$U_d$};
    \draw[->, thick, blue] ($(opOut) + (1.25,-0.1)$) -- +(0,-1.3) node[above right, blue] {$U_a$};
    \draw[->, thick, blue] (0,-0.1) -- +(0,-1.8) node[above right, blue] {$U_e$};


    \draw (0,0) circle (1.5pt);
	\draw (0,-2) circle (1.5pt); 
    \draw (5.5,-2) circle (1.5pt);
	\draw (5.5,-0.49) circle (1.5pt); 

    \filldraw[black]
        (opIn+) circle (1.5pt)
        (opIn-) circle (1.5pt)
        (opOut) circle (1.5pt);
\end{circuitikz}
\end{center}

\subsubsection{nicht invertierende Komparator}
\begin{center}
\begin{circuitikz}
    \coordinate (opAmp) at (0,0);
    \coordinate (opIn+) at ($(opAmp) + (-1.25,.5)$);
    \coordinate (opIn+Label) at ($(opIn+) + (-0.25,-0.1)$);
    \coordinate (opIn-) at ($(opAmp) + (-1.25,-.5)$);
    \coordinate (opIn-Label) at ($(opIn-) + (-0.25,0.1)$);
    \coordinate (opOut) at ($(opAmp) + (1.25,-0.1)$);

    \draw(opAmp) node[op amp, yscale=-1] {};
    \draw (1.25,-1.15) node[rground]{};


        
    \draw[->, thick, blue] (opIn+Label) -- (opIn-Label) node[above left, blue] {$U_D$};
    \draw[->, thick, blue] (opOut) -- +(0,-0.9) node[above right, blue] {$U_a$};

    \draw (-1,0.49) -- (-1.45,0.49);
    \draw (-1,-0.49) -- (-1.45,-0.49);

    \draw (1.25,0) circle (1.5pt);
	\draw (1.25,-1.1) circle (1.5pt); 
    \draw (-1.5,0.49) circle (1.5pt);
	\draw (-1.5,-0.49) circle (1.5pt); 
\end{circuitikz}
\end{center}
Der Minus-Eingang vom OPV wird als U$_c$ angenommen.\\
Es wird überprüft, ob Ue < oder > Uc ist.   
\begin{align}
    U_d > 0	=> U_a=HIGH \\
    U_d < 0	=> U_a=LOW
\end{align}



\subsubsection{invertierende Komparator}
\begin{center}
\begin{circuitikz}
    
    \coordinate (opIn+) at ($(opAmp) + (-1.25,.5)$);
    \coordinate (opIn+Label) at ($(opIn+) + (-0.25,-0.1)$);
    \coordinate (opIn-) at ($(opAmp) + (-1.25,-.5)$);
    \coordinate (opIn-Label) at ($(opIn-) + (-0.25,0.1)$);
    \coordinate (opOut) at ($(opAmp) + (1.25,-0.1)$);

    \node[op amp] at (0,0) (opamp) {};
    \draw (1.25,-1.15) node[rground]{};


        
    \draw[->, thick, blue] (opIn+Label) -- (opIn-Label) node[above left, blue] {$U_D$};
    \draw[->, thick, blue] (opOut) -- +(0,-0.9) node[above right, blue] {$U_a$};

    \draw (-1,0.49) -- (-1.45,0.49);
    \draw (-1,-0.49) -- (-1.45,-0.49);

    \draw (1.25,0) circle (1.5pt);
	\draw (1.25,-1.1) circle (1.5pt); 
    \draw (-1.5,0.49) circle (1.5pt);
	\draw (-1.5,-0.49) circle (1.5pt); 
\end{circuitikz}
\end{center}
Der Minus-Eingang vom OPV wird als U$_c$ angenommen.\\
Es wird überprüft, ob Ue < oder > Uc ist.

\begin{align}
Ud > 0	=> Ua=LOW\\
Ud < 0	=> Ua=HIGH
\end{align}


\subsection{Invertierender Verstärker}

\begin{center}
\begin{circuitikz}
        % \coordinate (opAmp) at (4,-0.5);
        % \coordinate (opIn+) at ($(opAmp) + (-1.25,.5)$);
        % \coordinate (opIn+Label) at ($(opIn+) + (-0.25,0)$);
        % \coordinate (opIn-) at ($(opAmp) + (-1.25,-.5)$);
        % \coordinate (opIn-Label) at ($(opIn-) + (-0.25,0)$);
        % \coordinate (opOut) at ($(opAmp) + (1.25,0)$);
    
        % % OpAmp
        % \draw(opAmp) node[op amp] {};
        
        % % Resistors
        % \draw(0,0) to[R, l=$R_1$] (opIn+);
        % \draw($(opIn+) + (0,1.5)$) to[R, l=$R_2$] ($(opIn+) + (2.5,1.5)$);
 
        % % Wires
        % \draw[black] (opOut) -- +(1,0);
        % \draw[black] (opIn+) -- ($(opIn+) + (0,1.5)$);
        % \draw[black] ($(opIn+) + (2.5,1.5)$) -- (opOut);

        % \draw[black] (opIn-) -- +(0,-1);
        
        % % Ground
        % \draw[ultra thick, black] ($(opIn-) +(-0.25,-1)$) -- +(0.5,0);

        % % Voltages
        % \draw[->, thick, blue] (opIn+Label) -- (opIn-Label) node[above left, blue] {$U_d$};
        \draw node [op amp](A1){};
        \draw(-4,0.49) to[R, l=$R_1$] (-2,0.49);
        \draw(-1,2) to[R, l=$R_2$] (1,2);
        \draw (-4.55,-1.55) node[rground]{};
        \draw (-2,-1.5) node[rground]{};
        \draw (2.55,-1.55) node[rground]{};

        \draw (-4,0.49) -- (-4.5,0.49);
        \draw (-2,-0.49) -- (-1,-0.49);
        \draw (-2,-0.49) -- (-2,-1.5);
        \draw (-2,0.49) -- (-1,0.49);
        \draw (-1.5,0.49) -- (-1.5,2);
        \draw (-1.5,2) -- (-1,2);
        \draw (1,2) -- (1.5,2);
        \draw (1.5,2) -- (1.5,0);
        \draw (1,0) -- (2.5,0);

        \draw (-4.55,0.49) circle (1.5pt);
    	\draw (-4.55,-1.5) circle (1.5pt); 
        \draw (2.55,0) circle (1.5pt);
    	\draw (2.55,-1.5) circle (1.5pt); 
        \draw[black,fill=black] (-1.5,0.49) circle (1.5pt);
    	\draw[black,fill=black] (1.5,0) circle (1.5pt);

        \draw[->, thick, blue] (-1.2,0.4) -- (-1.2,-0.4) node[above left, blue] {$U_d$};
        \draw[->, thick, blue] (-4.55,0.4) -- (-4.55,-1.4) node[above left, blue] {$U_e$};
        \draw[->, thick, blue] (2.55,-0.1) -- (2.55,-1.4) node[above left, blue] {$U_a$};
        \draw[<-, thick, blue] (-1.2,2.7) -- (1.5,2.7) node[above left, blue] {$U_{2}$};
        
\end{circuitikz}
\end{center}

$R_e$ ist der Eingangswiderstand der Schaltung.\\

Hier gilt: $R_e=R_1$

\begin{align}
    V = \frac{U_a}{U_e}=-\frac{R_2}{R_1}
\end{align}

\subsection{Nicht-Invertierender Verstärker}
\begin{center}
\begin{circuitikz}
        % \coordinate (opAmp) at (2,-0.5);
        % \coordinate (opIn+) at ($(opAmp) + (-1.25,.5)$);
        % \coordinate (opIn+Label) at ($(opIn+) + (-0.25,0)$);
        % \coordinate (opIn-) at ($(opAmp) + (-1.25,-.5)$);
        % \coordinate (opIn-Label) at ($(opIn-) + (-0.25,0)$);
        % \coordinate (opOut) at ($(opAmp) + (1.25,0)$);
    
        % % OpAmp
        % \draw(opAmp) node[op amp, yscale=-1] {};

        % % Resistors
        % \draw(opOut) to[R, l=$R_1$] ($(opOut) + (0,-2)$);
        % \draw($(opOut) + (0,-2)$) to[R, l=$R_2$] ($(opOut) + (0,-4)$);

        % % Wires
        % \draw[black] (opOut) -- +(1,0);
        % \draw[black] (opIn+) -- ($(opIn+) + (-1,0)$);
        % \draw[black] (opIn-) -- ($(opIn+) + (0,-2.5)$) -- ($(opOut) + (0,-2)$);

        % % Voltages
        % \draw[->, thick, blue] (opIn+Label) -- (opIn-Label) node[above left, blue] {$U_d$};
        \node[op amp, noinv input up] at (0,0) (opamp) {};
        \draw(1.5,0) to[R, l=$R_1$] (1.5,-2);
        \draw(1.5,-2) to[R, l=$R_2$] (1.5,-4);
        \draw (-4.55,-4) node[rground]{};
        \draw (1.5,-4) node[rground]{};
        \draw (3.55,-4) node[rground]{};

        \draw (-2,-0.49) -- (-1,-0.49);
        \draw (-2,-0.49) -- (-2,-2);
        \draw (-2,-2) -- (1.5,-2);

        
        \draw (-1,0.49) -- (-4.5,0.49);
        \draw (1,0) -- (3.5,0);

        \draw (-4.55,0.49) circle (1.5pt);
    	\draw (-4.55,-3.95) circle (1.5pt); 
        \draw (3.55,0) circle (1.5pt);
    	\draw (3.55,-3.95) circle (1.5pt); 
        \draw[black,fill=black] (1.5,-2) circle (1.5pt);
    	\draw[black,fill=black] (1.5,0) circle (1.5pt);

        \draw[->, thick, blue] (-1.2,0.4) -- (-1.2,-0.4) node[above left, blue] {$U_d$};
        \draw[->, thick, blue] (-4.55,0.4) -- (-4.55,-3.8) node[above left, blue] {$U_e$};
        \draw[->, thick, blue] (3.55,-0.1) -- (3.55,-3.8) node[above left, blue] {$U_a$};
\end{circuitikz}
\end{center}

$R_e$ ist der Eingangswiderstand der Schaltung.\\

Hier gilt: $R_e=\infty$

\begin{align}
    V = \frac{U_a}{U_d}=\frac{R_1+R_2}{R_2}
\end{align}

\section{Schmitttrigger}

\subsection{Invertierender Schmitttrigger}
\begin{align}
    \frac{U_a}{U_e}=\frac{R_1+R_2}{R_2}
\end{align}

\begin{center}
\begin{circuitikz}
    % \coordinate (opAmp) at (2,-0.5);
    % \coordinate (opIn+) at ($(opAmp) + (-1.25,.5)$);
    % \coordinate (opIn+Label) at ($(opIn+) + (-0.25,0)$);
    % \coordinate (opIn-) at ($(opAmp) + (-1.25,-.5)$);
    % \coordinate (opIn-Label) at ($(opIn-) + (-0.25,0)$);
    % \coordinate (opOut) at ($(opAmp) + (1.25,0)$);

    % % OpAmp
    % \draw(opAmp) node[op amp] {};

    % % Resistors
    % \draw(opOut) to[R, l=$R_1$] ($(opOut) + (0,-2)$);
    % \draw($(opOut) + (0,-2)$) to[R, l=$R_2$] ($(opOut) + (0,-4)$);

    % % Wires
    % \draw[black] (opOut) -- +(1,0);
    % \draw[black] (opIn+) -- ($(opIn+) + (-1,0)$);
    % \draw[black] (opIn-) -- ($(opIn+) + (0,-2.5)$) -- ($(opOut) + (0,-2)$);

    % % Voltages
    % \draw[->, thick, blue] (opIn+Label) -- (opIn-Label) node[above left, blue] {$U_d$};
        \node[op amp] at (0,0) (opamp) {};
        \draw(1.5,0) to[R, l=$R_1$] (1.5,-2);
        \draw(1.5,-2) to[R, l=$R_2$] (1.5,-4);
        \draw (-4.55,-4) node[rground]{};
        \draw (1.5,-4) node[rground]{};
        \draw (3.55,-4) node[rground]{};

        \draw (-2,-0.49) -- (-1,-0.49);
        \draw (-2,-0.49) -- (-2,-2);
        \draw (-2,-2) -- (1.5,-2);

        
        \draw (-1,0.49) -- (-4.5,0.49);
        \draw (1,0) -- (3.5,0);

        \draw (-4.55,0.49) circle (1.5pt);
    	\draw (-4.55,-3.95) circle (1.5pt); 
        \draw (3.55,0) circle (1.5pt);
    	\draw (3.55,-3.95) circle (1.5pt); 
        \draw[black,fill=black] (1.5,-2) circle (1.5pt);
    	\draw[black,fill=black] (1.5,0) circle (1.5pt);

        \draw[->, thick, blue] (-1.2,0.4) -- (-1.2,-0.4) node[above left, blue] {$U_d$};
        \draw[->, thick, blue] (-4.55,0.4) -- (-4.55,-3.8) node[above left, blue] {$U_e$};
        \draw[->, thick, blue] (3.55,-0.1) -- (3.55,-3.8) node[above left, blue] {$U_a$};
\end{circuitikz}
\end{center}

\subsection{Nicht-Invertierender Schmitttrigger}
\begin{align}
    \frac{U_a}{U_e}=\frac{R_2}{R_1}
\end{align}

\begin{center}
\begin{circuitikz}
        % \coordinate (opAmp) at (4,-0.5);
        % \coordinate (opIn+) at ($(opAmp) + (-1.25,.5)$);
        % \coordinate (opIn+Label) at ($(opIn+) + (-0.25,0)$);
        % \coordinate (opIn-) at ($(opAmp) + (-1.25,-.5)$);
        % \coordinate (opIn-Label) at ($(opIn-) + (-0.25,0)$);
        % \coordinate (opOut) at ($(opAmp) + (1.25,0)$);
    
        % % OpAmp
        % \draw(opAmp) node[op amp, yscale=-1] {};
        
        % % Resistors
        % \draw(0,0) to[R, l=$R_1$] (opIn+);
        % \draw($(opIn+) + (0,1.5)$) to[R, l=$R_2$] ($(opIn+) + (2.5,1.5)$);
    
        % % Wires
        % \draw[black] (opOut) -- +(1,0);
        % \draw[black] (opIn+) -- ($(opIn+) + (0,1.5)$);
        % \draw[black] ($(opIn+) + (2.5,1.5)$) -- (opOut);

        % \draw[black] (opIn-) -- +(0,-1);
        
        % % Ground
        % \draw[ultra thick, black] ($(opIn-) +(-0.25,-1)$) -- +(0.5,0);

        % % Voltages
        % \draw[->, thick, blue] (opIn+Label) -- (opIn-Label) node[above left, blue] {$U_d$};
        \draw node [op amp,  noinv input up](A1){};
        \draw(-4,0.49) to[R, l=$R_1$] (-2,0.49);
        \draw(-1,2) to[R, l=$R_2$] (1,2);
        \draw (-4.55,-1.55) node[rground]{};
        \draw (-2,-1.5) node[rground]{};
        \draw (2.55,-1.55) node[rground]{};

        \draw (-4,0.49) -- (-4.5,0.49);
        \draw (-2,-0.49) -- (-1,-0.49);
        \draw (-2,-0.49) -- (-2,-1.5);
        \draw (-2,0.49) -- (-1,0.49);
        \draw (-1.5,0.49) -- (-1.5,2);
        \draw (-1.5,2) -- (-1,2);
        \draw (1,2) -- (1.5,2);
        \draw (1.5,2) -- (1.5,0);
        \draw (1,0) -- (2.5,0);

        \draw (-4.55,0.49) circle (1.5pt);
    	\draw (-4.55,-1.5) circle (1.5pt); 
        \draw (2.55,0) circle (1.5pt);
    	\draw (2.55,-1.5) circle (1.5pt); 
        \draw[black,fill=black] (-1.5,0.49) circle (1.5pt);
    	\draw[black,fill=black] (1.5,0) circle (1.5pt);

        \draw[->, thick, blue] (-1.2,0.4) -- (-1.2,-0.4) node[above left, blue] {$U_d$};
        \draw[->, thick, blue] (-4.55,0.4) -- (-4.55,-1.4) node[above left, blue] {$U_e$};
        \draw[->, thick, blue] (2.55,-0.1) -- (2.55,-1.4) node[above left, blue] {$U_a$};
\end{circuitikz}
\end{center}

\newpage

\section{Addierer}
Wenn $R_1=R_2=R_3$ dann gilt $U_a=-(U_{R_1}+U_{R_2})$.
\begin{center}
\begin{circuitikz}
        \draw node [op amp](A1){};
        \draw(-4,0.49) to[R, l=$R_1$] (-2,0.49);
        \draw(-4,2) to[R, l=$R_3$] (-2,2);
        \draw(-1,2) to[R, l=$R_2$] (1,2);
        \draw (-6.05,-1.55) node[rground]{};
        \draw (-4.55,-1.55) node[rground]{};
        \draw (-2,-1.5) node[rground]{};
        \draw (2.55,-1.55) node[rground]{};

        \draw (-2,2) -- (-1,2);
        \draw (-4,2) -- (-6,2);
        \draw (-4,0.49) -- (-4.5,0.49);
        \draw (-2,-0.49) -- (-1,-0.49);
        \draw (-2,-0.49) -- (-2,-1.5);
        \draw (-2,0.49) -- (-1,0.49);
        \draw (-1.5,0.49) -- (-1.5,2);
        \draw (-1.5,2) -- (-1,2);
        \draw (1,2) -- (1.5,2);
        \draw (1.5,2) -- (1.5,0);
        \draw (1,0) -- (2.5,0);

        \draw (-6.05,2) circle (1.5pt);
    	\draw (-6.05,-1.5) circle (1.5pt);     
        \draw (-4.55,0.49) circle (1.5pt);
    	\draw (-4.55,-1.5) circle (1.5pt); 
        \draw (2.55,0) circle (1.5pt);
    	\draw (2.55,-1.5) circle (1.5pt); 
        \draw[black,fill=black] (-1.5,0.49) circle (1.5pt);
    	\draw[black,fill=black] (1.5,0) circle (1.5pt);
        \draw[black,fill=black] (-1.5,2) circle (1.5pt);

        \draw[->, thick, blue] (-1.2,0.4) -- (-1.2,-0.4) node[above left, blue] {$U_d$};
        \draw[->, thick, blue] (-6.05,1.9) -- (-6.05,-1.4) node[above left, blue] {$U_{e2}$};
        \draw[->, thick, red] (-4.55,0.4) -- (-4.55,-1.4) node[above left, red] {$U_{e1}$};
        \draw[->, thick, blue] (2.55,-0.1) -- (2.55,-1.4) node[above left, blue] {$U_a$};
\end{circuitikz}
\end{center}

\subsubsection*{Berechnung mit Teilströmen}
\begin{align}
    I_1&=\frac{U_{e_1}}{R_1} \\
    I_2&=\frac{U_{e_2}}{R_2}
\end{align}
\begin{align}   
    U_{R_g}&=R_g\cdot(I_1+I_3) \\
    U_{R_g}&=R_g\cdot(\frac{U_{e_1}}{R_1}+\frac{U_{e_2}}{R_2}) \\
    U_{R_g}&=R_g\cdot(\frac{U_{e_1}\cdot R_g}{R_1}+\frac{U_{e_2}\cdot R_g}{R_2})
\end{align}
\todo{Woher das Minus auf einmal?}

\subsubsection*{Berechnung mit Überlagerungsprinzip}

\underline{$U_{e_1}$ wirkt, $U_{e_2}=0$:} \hspace{2cm} $U_a'=\frac{R_g}{R_1}\cdot U_{e_1}$ \\

\underline{$U_{e_2}$ wirkt, $U_{e_2}=1$:} \hspace{2cm} $U_a''=\frac{R_g}{R_2}\cdot U_{e_2}$ \\

\underline{Gesamt:}
\begin{align}  
    U_a&=U_a'+U_a''  \\
    U_a&=\frac{R_g}{R_1}\cdot U_{e_1}+\frac{R_g}{R_2}\cdot U_{e_2} \\
    U_a&=-(U_{e_1}\cdot\frac{R_g}{R_1}+U_{e_2}\cdot\frac{R_g}{R_2})
\end{align}

\section{Subtrahierer}
\subsection{Typ 1}
\begin{center}
\begin{circuitikz}
        \draw node [op amp](A1){};
        \draw(-4,0.49) to[R, l=$R_1$] (-2,0.49);
        \draw(-1,2) to[R, l=$R_2$] (1,2);
        \draw(-4,-0.49) to[R, l=$R_1$] (-2,-0.49);
        \draw(-2,-0.49) to[R, l=$R_2$] (-2,-2.49);
        \draw (-5.55,-2.49) node[rground]{};
        \draw (-4.05,-2.49) node[rground]{};
        \draw (-2,-2.49) node[rground]{};
        \draw (2.55,-2.49) node[rground]{};

        \draw (-4,0.49) -- (-4.5,0.49);
        \draw (-2,-0.49) -- (-1,-0.49);
        
        \draw (-2,0.49) -- (-1,0.49);
        \draw (-4,0.49) -- (-5.5,0.49);
        \draw (-1.5,0.49) -- (-1.5,2);
        \draw (-1.5,2) -- (-1,2);
        \draw (1,2) -- (1.5,2);
        \draw (1.5,2) -- (1.5,0);
        \draw (1,0) -- (2.5,0);

        \draw (-5.55,0.49) circle (1.5pt);
    	\draw (-5.55,-2.49) circle (1.5pt);
        \draw (-4.05,-0.49) circle (1.5pt);
        \draw (-4.05,-2.49) circle (1.5pt);
        \draw (2.55,0) circle (1.5pt);
    	\draw (2.55,-2.49) circle (1.5pt); 
        \draw[black,fill=black] (-1.5,0.49) circle (1.5pt);
    	\draw[black,fill=black] (1.5,0) circle (1.5pt);

        \draw[->, thick, blue] (-1.2,0.4) -- (-1.2,-0.4) node[above left, blue] {$U_d$};
        \draw[->, thick, blue] (-5.55,0.4) -- (-5.55,-2.4) node[above left, blue] {$U_e2$};
        \draw[->, thick, red] (-4.05,-0.6) -- (-4.05,-2.4) node[above left, red] {$U_e1$};
        \draw[->, thick, blue] (2.55,-0.1) -- (2.55,-2.4) node[above left, blue] {$U_a$};
\end{circuitikz}
\end{center}

\begin{align}
    U_a=(U_{e_2}-U_{e_1}\cdot{\frac{R_2}{R_1}})
\end{align}

\subsection{Typ 2}
\begin{center}
\begin{circuitikz}
        \draw node [op amp](A1){};
        \draw(-4,0.49) to[R, l=$R_2$] (-2,0.49);
        \draw(-4,2) to[R, l=$R_2$] (-2,2);
        \draw(-1,2) to[R, l=$R_4$] (1,2);
        \draw(-4,-0.49) to[R, l=$R_1$] (-2,-0.49);
        \draw(-4,-1.49) to[R, l=$R_1$] (-2,-1.49);
        \draw(-2,-1.49) to[R, l=$R_3$] (-2,-3.49);
        \draw (-2,-3.49) node[rground]{};
        \draw (2.55,-2.49) node[rground]{};

        
        \draw (-2,-0.49) -- (-1,-0.49);
        \draw (-2,0.49) -- (-1,0.49);
        \draw (-1.5,0.49) -- (-1.5,2);
        \draw (-1.5,2) -- (-1,2);
        \draw (1,2) -- (1.5,2);
        \draw (1.5,2) -- (1.5,0);
        \draw (1,0) -- (2.5,0);
        \draw (-2,-1.49) -- (-2,-0.49);
        \draw (-1.5,2) -- (-2,2);

        \draw (-4.05,0.49) circle (1.5pt);
    	\draw (-4.05,2) circle (1.5pt);
        \draw (-4.05,-0.49) circle (1.5pt);
        \draw (-4.05,-1.49) circle (1.5pt);
        \draw (2.55,0) circle (1.5pt);
    	\draw (2.55,-2.49) circle (1.5pt); 
        \draw[black,fill=black] (-1.5,0.49) circle (1.5pt);
    	\draw[black,fill=black] (1.5,0) circle (1.5pt);
        \draw[black,fill=black] (-1.5,2) circle (1.5pt);
        \draw[black,fill=black] (-2,-0.49) circle (1.5pt);
        \draw[black,fill=black] (-2,-1.49) circle (1.5pt);

        \draw[ thick, red] (-3.8,2.7) -- (-2.2,2.7) node[above left, red] {};
        \draw[ thick, red] (-2.2,2.7) -- (-2.2,0.2) node[above left, red] {};
        \draw[ thick, red] (-2.2,0.2) -- (-3.8,0.2) node[above left, red] {};
        \draw[ thick, red] (-3.8,0.2) -- (-3.8,2.7) node[above left, red] {$A$}; 

        \draw[ thick, green] (-3.8,0.15) -- (-2.2,0.15) node[above left, green] {};
        \draw[ thick, green] (-2.2,0.15) -- (-2.2,-2) node[above left, green] {};
        \draw[ thick, green] (-2.2,-2) -- (-3.8,-2) node[below left, green] {$B$};
        \draw[ thick, green] (-3.8,-2) -- (-3.8,0.15) node[above left, green] {}; 

        \draw[->, thick, blue] (-1.2,0.4) -- (-1.2,-0.4) node[above left, blue] {$U_d$};
        \draw[->, thick, blue] (2.55,-0.1) -- (2.55,-2.4) node[above left, blue] {$U_a$};
\end{circuitikz}
\end{center}
\begin{align}
    U_a=(\sum B - \sum A)\cdot\frac{R_2}{R_1}
\end{align}

\subsection{Typ 3}
\begin{center}
\begin{circuitikz}
        \draw node [op amp](A1){};
        \draw(-4,0.49) to[R, l=$R_2$] (-2,0.49);
        \draw(-1,2) to[R, l=$R_4$] (1,2);
        \draw(-4,-0.49) to[R, l=$R_1$] (-2,-0.49);
        \draw(-2,-0.49) to[R, l=$R_3$] (-2,-2.49);
        \draw (-5.55,-2.49) node[rground]{};
        \draw (-4.05,-2.49) node[rground]{};
        \draw (-2,-2.49) node[rground]{};
        \draw (2.55,-2.49) node[rground]{};

        \draw (-4,0.49) -- (-4.5,0.49);
        \draw (-2,-0.49) -- (-1,-0.49);
        
        \draw (-2,0.49) -- (-1,0.49);
        \draw (-4,0.49) -- (-5.5,0.49);
        \draw (-1.5,0.49) -- (-1.5,2);
        \draw (-1.5,2) -- (-1,2);
        \draw (1,2) -- (1.5,2);
        \draw (1.5,2) -- (1.5,0);
        \draw (1,0) -- (2.5,0);

        \draw (-5.55,0.49) circle (1.5pt);
    	\draw (-5.55,-2.49) circle (1.5pt);
        \draw (-4.05,-0.49) circle (1.5pt);
        \draw (-4.05,-2.49) circle (1.5pt);
        \draw (2.55,0) circle (1.5pt);
    	\draw (2.55,-2.49) circle (1.5pt); 
        \draw[black,fill=black] (-1.5,0.49) circle (1.5pt);
    	\draw[black,fill=black] (1.5,0) circle (1.5pt);

        \draw[->, thick, blue] (-1.2,0.4) -- (-1.2,-0.4) node[above left, blue] {$U_d$};
        \draw[->, thick, blue] (-5.55,0.4) -- (-5.55,-2.4) node[above left, blue] {$U_e2$};
        \draw[->, thick, red] (-4.05,-0.6) -- (-4.05,-2.4) node[above left, red] {$U_e1$};
        \draw[->, thick, blue] (2.55,-0.1) -- (2.55,-2.4) node[above left, blue] {$U_a$};
\end{circuitikz}
\end{center}
\underline{Berechnung mit Überlagerungsprinzip} \\
\underline{$U_{e_1}$ wirkt, $U_{e_2}=0$:}
\begin{align}
    U_a'=\frac{R_4+R_2}{R_2}\cdot U_{e_1}\cdot\frac{R_3}{R_1+R_3}
\end{align}

\underline{$U_{e_2}$ wirkt, $U_{e_2}=1$:}
\begin{align}
    U_a''=-U_{e_2}\cdot \frac{R_4}{R_2}
\end{align}

\underline{Gesamt}
\begin{align}
    U_a&=U_a'+U_a'' \\
    U_a&=U_{e_1}\cdot (\frac{R_4+R_2}{R_2}\cdot\frac{R_3}{R_1+R_3})-U_{e_2}\cdot\frac{R_4}{R_2}
\end{align}

Wenn alle $R$ gleich groß sind, gilt: $U_a=U_{e_1}-U_{e_2}$

\todo{Berechnung mit Teilspannungen}

\section{Integrator \and Differentiator}
\begin{center}
\begin{circuitikz}
        \draw node [op amp](A1){};
        \draw(-4,0.49) to[R, l=$R_1$] (-2,0.49);
        \draw(-1,2) to[R, l=$R_2$] (1,2);
        \draw(-1,3.5) to[C, l=$C$] (1,3.5);
        \draw (-4.55,-1.55) node[rground]{};
        \draw (-2,-1.5) node[rground]{};
        \draw (2.55,-1.55) node[rground]{};

        \draw (-4,0.49) -- (-4.5,0.49);
        \draw (-2,-0.49) -- (-1,-0.49);
        \draw (-2,-0.49) -- (-2,-1.5);
        \draw (-2,0.49) -- (-1,0.49);
        \draw (-1.5,0.49) -- (-1.5,2);
        \draw (-1.5,2) -- (-1,2);
        \draw (1,2) -- (1.5,2);
        \draw (1.5,2) -- (1.5,0);
        \draw (1,0) -- (2.5,0);
        \draw (-1.5,2) -- (-1.5,3.5);
        \draw (-1.5,3.5) -- (-1,3.5);
        \draw (1.5,3.5) -- (1,3.5);
        \draw (1.5,2) -- (1.5,3.5);

        \draw (-4.55,0.49) circle (1.5pt);
    	\draw (-4.55,-1.5) circle (1.5pt); 
        \draw (2.55,0) circle (1.5pt);
    	\draw (2.55,-1.5) circle (1.5pt); 
        \draw[black,fill=black] (-1.5,0.49) circle (1.5pt);
    	\draw[black,fill=black] (1.5,0) circle (1.5pt);

        \draw[->, thick, blue] (-1.2,0.4) -- (-1.2,-0.4) node[above left, blue] {$U_d$};
        \draw[->, thick, blue] (-4.55,0.4) -- (-4.55,-1.4) node[above left, blue] {$U_e$};
        \draw[->, thick, blue] (2.55,-0.1) -- (2.55,-1.4) node[above left, blue] {$U_a$};
\end{circuitikz}
\end{center}
Differentiator:
\begin{center}
\begin{circuitikz}
        \draw node [op amp](A1){};
        \draw(-6,0.49) to[R, l=$R_1$] (-4,0.49);
        \draw(-1,2) to[R, l=$R_2$] (1,2);
        \draw(-4,0.49) to[C, l=$C_1$] (-2,0.49);
        \draw (-6.05,-1.55) node[rground]{};
        \draw (-2,-1.5) node[rground]{};
        \draw (2.55,-1.55) node[rground]{};

        
        \draw (-2,-0.49) -- (-1,-0.49);
        \draw (-2,-0.49) -- (-2,-1.5);
        \draw (-2,0.49) -- (-1,0.49);
        \draw (-1.5,0.49) -- (-1.5,2);
        \draw (-1.5,2) -- (-1,2);
        \draw (1,2) -- (1.5,2);
        \draw (1.5,2) -- (1.5,0);
        \draw (1,0) -- (2.5,0);

        \draw (-6.05,0.49) circle (1.5pt);
    	\draw (-6.05,-1.5) circle (1.5pt); 
        \draw (2.55,0) circle (1.5pt);
    	\draw (2.55,-1.5) circle (1.5pt); 
        \draw[black,fill=black] (-1.5,0.49) circle (1.5pt);
    	\draw[black,fill=black] (1.5,0) circle (1.5pt);

        \draw[->, thick, blue] (-1.2,0.4) -- (-1.2,-0.4) node[above left, blue] {$U_d$};
        \draw[->, thick, blue] (-6.05,0.4) -- (-6.05,-1.4) node[above left, blue] {$U_e$};
        \draw[->, thick, blue] (2.55,-0.1) -- (2.55,-1.4) node[above left, blue] {$U_a$};
        
        
\end{circuitikz}
\end{center}

\section{Pegelwandler}
\begin{center}
\begin{circuitikz}
        \draw node [op amp,  noinv input up](A1){};
        \draw(-4,0.49) to[R, l=$R_1$] (-2,0.49);
        \draw(-1,2) to[R, l=$R_2$] (1,2);
        \draw (-4.55,-1.55) node[rground]{};
        \draw (-2,-1.5) node[rground]{};
        \draw (2.55,-1.55) node[rground]{};

        \draw (-4,0.49) -- (-4.5,0.49);
        \draw (-2,-0.49) -- (-1,-0.49);
        \draw (-2,-0.49) -- (-2,-1.5);
        \draw (-2,0.49) -- (-1,0.49);
        \draw (-1.5,0.49) -- (-1.5,2);
        \draw (-1.5,2) -- (-1,2);
        \draw (1,2) -- (1.5,2);
        \draw (1.5,2) -- (1.5,0);
        \draw (1,0) -- (2.5,0);

        \draw (-4.55,0.49) circle (1.5pt);
    	\draw (-4.55,-1.5) circle (1.5pt); 
        \draw (2.55,0) circle (1.5pt);
    	\draw (2.55,-1.5) circle (1.5pt); 
        \draw[black,fill=black] (-1.5,0.49) circle (1.5pt);
    	\draw[black,fill=black] (1.5,0) circle (1.5pt);

        \draw[->, thick, blue] (-1.2,0.4) -- (-1.2,-0.4) node[above left, blue] {$U_d$};
        \draw[->, thick, blue] (-4.55,0.4) -- (-4.55,-1.4) node[above left, blue] {$U_e$};
        \draw[->, thick, blue] (2.55,-0.1) -- (2.55,-1.4) node[above left, blue] {$U_a$};
        \draw[->, thick, blue] (-4.55,3.5) -- (1.5,3.5) node[above left, blue] {$U_{abfall}$};
        \draw[->, thick, blue] (-1.2,2.7) -- (1.5,2.7) node[above left, blue] {$U_{2}$};
\end{circuitikz}
\end{center}
Hier gilt:
\begin{align}
    V=-\frac{R_2}{R_1}=\frac{\Delta U_a}{\Delta U_e}
\end{align}

\subsubsection*{Beispiel}
Das Eingangssignal von $U_e=-1V$ bis $+1V$ soll am Ausgang zu $U_a=0V$ bis $+5V$ gewandelt werden.
\begin{align}
    &V=-\frac{R_2}{R_1}=\frac{\Delta U_a}{\Delta U_e}   \\
    &V=-\frac{5V-0V}{1V-(-1V)}=\frac{5V}{2V}            \\
    &V=-\frac{5k\Omega}{2k\Omega}
\end{align}

\begin{align}
    \frac{U_{R_2}}{U_e-U_a}&=\frac{R_2}{R_1+R_2}                \\
    U_{R_2}&=(U_e-U_a)-\frac{R_2}{R_1+R_2}                      \\
    U_{R_2}&=(1V-0V)-\frac{5k\Omega}{2k\Omega+5k\Omega}         \\
    U_{R_2}&\approx 0,715V                                      \\
    \Rightarrow U_+=U_a+U_{R_2}&=0V+0,714V=0,714V
\end{align}

\section{Instrumentation-Amplifier}
\begin{center}
\begin{circuitikz}
        \draw node [op amp](A1) at (0,-3) {}  ;
        \draw node [op amp,  noinv input up]() at (0,5){};
        \draw node [op amp](A1) at (6,1){};
        \draw (1.2,-1) to[vR, l=$R_{gain}$] ++(0,4);
        \draw(1.2,5) to[R, l=$R_2$] (1.2,3);
        \draw(1.2,-1) to[R, l=$R_2$] (1.2,-3);
        \draw(1.2,5) to[R, l=$R_3$] (3.2,5);
        \draw(1.2,-3) to[R, l=$R_3$] (3.2,-3);
        \draw(4.8,5) to[R, l=$R_4$] (7.2,5);
        \draw(4.8,-3) to[R, l=$R_4$] (7.2,-3);
        
        \draw (-2.05,3) node[rground]{};
        \draw (-2.05,-5) node[rground]{};
        \draw (7.2,-3) node[rground]{};
        \draw (8.55,-3) node[rground]{};

        \draw (-1.2,4.5) -- (-1.2,3);
        \draw (-1.2,3) -- (1.2,3);
        \draw (-1.2,-2.5) -- (-1.2,-1);
        \draw (-1.2,-1) -- (1.2,-1);
        \draw (3,5) -- (4.8,5);
        \draw (4.8,5) -- (4.8,1.5);
        \draw (3,-3) -- (4.8,-3);
        \draw (4.8,-3) -- (4.8,0.5);
        \draw (7.2,5) -- (7.2,1);
        \draw (7.2,1) -- (8.5,1);
        \draw (-2,5.5) -- (-1,5.5);
        \draw (-2,-3.5) -- (-1,-3.5);


        \draw (-2.05,5.5) circle (1.5pt);
        \draw (-2.05,3.05) circle (1.5pt); 
        \draw (-2.05,-3.5) circle (1.5pt);
    	\draw (-2.05,-4.95) circle (1.5pt);
    	\draw (8.55,1) circle (1.5pt);
        \draw (8.55,-2.95) circle (1.5pt);
     
        \draw[black,fill=black] (7.2,1) circle (1.5pt);
    	\draw[black,fill=black] (1.2,5) circle (1.5pt);
        \draw[black,fill=black] (1.2,3) circle (1.5pt);
    	
        \draw[black,fill=black] (1.2,-1) circle (1.5pt);
    	\draw[black,fill=black] (4.8,5) circle (1.5pt);
        \draw[black,fill=black] (4.8,-3) circle (1.5pt);
    	\draw[black,fill=black] (1.2,-3) circle (1.5pt);
     
        \draw[->, thick, blue] (-2.05,5.4) -- (-2.05,3.15) node[above left, blue] {$U_1$};
        \draw[->, thick, blue] (-2.05,-3.6) -- (-2.05,-4.85) node[above left, blue] {$U_2$};
        \draw[->, thick, blue] (8.55,0.9) -- (8.55,-2.85) node[above left, blue] {$U_a$};
\end{circuitikz}
\end{center}

    \chapter{Simulation}

\section{Altium}
Es muss beachtet werden, dass nicht alle BAuteile simmuliert werden können, sondern nur die der "Simulation Generic Components" Library.\\

Außerdem ist es besonders wichtig, dass bei \textbf{OPV-Schaltungen} die einen \textbf{Verstärker} implementieren, der \textbf{OPV} benutzt werden muss.\\
Bei Schaltungen die den OPV als \textbf{Schmitttrigger} (oder etwas Ähnliches) verwenden, muss der Komperator verwendet werden.

\subsection{Quellen}
Ab AltiumDesigner21 sind alle Quellen gleich; unter \verb|Simulate| $\rightarrow$ \verb|Sources| können Quellen platziert werden:

\subsection{Probes}
Es können unter \verb|Simulate| $\rightarrow$ \verb|Place Probes| Probes platziert werden, mit denen Spannungen und Ströme gemessen werden können. In AD17 war bzw. ist dies mit Netlabels möglich.

\subsection{Simulation}
Unter \verb|Simulate| $\rightarrow$ \verb|Simulation Dashboard| kann die Simulation eingestellt und gestartet werden. \\
Zunächst wird die Schaltung verifiziert; um die Verifikation zu aktualisieren, muss im Drop-Down-Menü ein anderes Dokument ausgewählt werden und dann zurückgewechselt werden. \\

Danach werden die Quellen und Probes ausgewählt: Mit einem Häkchen können diese entsprechend deaktiviert werden. Mit dem gefärbten Kästchen kann die Farbe der Probe im Plot verändert werden. \\

Als Nächstes wird die Simulation eingestellt, wobei \verb|Transient| und \verb|AC-Sweep| besonders wichtig sind:
\begin{itemize}
    \item \underline{Transient-Analyse} \\
    Bei der Transienten-Analyse wird die Schaltung in einem bestimmen Zeitbereich simuliert. Dieser kan sowohl in Zeit, als auch in Perioden angegeben werden (wobei diese nur bei Quellen mit periodischen Signalen funktioniert). Die zuvor ausgewählten Probes werden automatisch in einem Plot angezeigt. Mit dem "\verb|+ Add|"-Knopf können weiter Plots bzw. Signale hinzugefügt werden (siehe \todo{Add Reference}). \\
    Mit dem Häkchen "\verb|Use Initial Conditions|" können Anfangsspannungen festgelegt werden, welche in der Schematik über \verb|Simulate| $\rightarrow$ \verb|Place Initial Condition| platziert werden. \\
    In den Einstellungen der \verb|Initial Condition| (erreichbar via Doppelklick) kann unter \verb|Parameters| die Spannung eingstellt werden.

    \item \underline{AC-Sweep} \\
    Beim AC-Sweep wird die Frequenz der angegebenen Quellen in einem festgelegten Bereich "gesweeped", d.h. vom Minimum zum Maximum schrittweise durchgerechnet. Die Frequenz kann in Dekaden (logarithmisch), Okataven oder Linear angegeben werden. \\

    Mit dem "\verb|+ Add|"-Knopf können zusätzliche Outputs hinzugefügt werden, wodurch ein neues Feld erscheint. In diesem kann über den Drei-Punkte-Knopf der Output konfiguriert werden.\\

    In \verb|Waveforms| stehen alle möglichen Signale, welche in der Schaltung vorkommen (Probes, Netlabels, Widerstandsspannungen, etc.) Probes erhalten den Namen folgendermaßen: \verb|v("Net", Bauteil, "_", Probe Nummer)|. Eine Probe mit der Nummer $1$, an einem Widerstand, würde dementsprechend \verb|v(NetR1_1)| heißen. \\

    In \verb|Functions| stehen alle möglichen Operationen wie Addition, Umrechnen in dB oder Berechnung der Phase. In Expression-X/Y steht was wirklich angezeigt werden soll. \\
    Beachte dass es das Expression-X-Feld in AD21 nicht gibt.

    \subsubsection*{Beispiel}
    Bodediagramm eines Tiefpasses.

    \begin{itemize}
        \item \underline{Übertragungsfunktion}: Expression-Y: \verb|db(v(NetC1_2/v(NetR1_1)))|
        \item \underline{Phasengang}: Expression-Y: \verb|PHASE(v(NetC1_2))|
    \end{itemize}
    \verb|Net1| ist der Eingang des Tiefpasses, \verb|Net2| der Ausgang.

    \underline{Simulation Dashboard}
    \underline{Simulation Output}

    Sollte eine Achse fehlen: doppelklicken auf die Achse und unter "Label" die gewünschte Einheit eintragen. \\

    Die entsprechende Schematik des Beispiels:
\end{itemize}

\newpage

\section{MicroCap}
\subsection{Komponentenauswahl}
\begin{figure}[h]
    \centering
    % \includegraphics[width=0.7\linewidth]{Microcap-Bilder/Bauteileaussuchen.PNG}
    \caption{Komponentenauswahl}
    \label{fig:Komponentenauswahl}
\end{figure}

\subsubsection*{Komponenten suchen}
$\rightarrow$ Components $\rightarrow$ Find Component oder:\\
in linkem Fenster $\rightarrow$ Search
\begin{figure}[h]
    \centering
    % \includegraphics[width=0.7\linewidth]{Microcap-Bilder/Bauteilesuchen.PNG}
    \caption{Komponenten suchen}
    \label{fig:Komponenten suchen}
\end{figure}

\newpage

\subsection{Bauteile verbinden}
$\rightarrow$ Wire Mode $\rightarrow$ linke Maustaste gedrückt halten um Verbindungen zu ziehen
\begin{figure}[h]
    \centering
    % \includegraphics[width=1\linewidth]{Microcap-Bilder/Bauteileverbinden.PNG}
    \caption{Bauteile verbinden}
    \label{fig:Bauteile verbinden}
\end{figure}

\subsection{Bauteile konfigurieren}
jedes Bauteil muss individuell konfiguriert werden $\rightarrow$ auf Syntax achten wenn unklar\\
mit "Plot" können Diagramme angezeigt werden (z.Bsp. bei Pulse Source das eingestellte Signal im Zeitbereich, beim Kondensator die Impedanz im Vergleich zur Frequenz, ...)

\subsubsection*{Werte}
Microcap unterscheidet nicht zwischen Groß-und Kleinschreibung!
$\rightarrow$ 1M = 1m = 1 milli
\begin{itemize}
    \item Mega $\rightarrow$ meg
    \item Kilo $\rightarrow$ k
    \item Milli $\rightarrow$ m
    \item µ $\rightarrow$ u
    \item Nano $\rightarrow$ n
\end{itemize}

\subsection*{Sinussignal}
\begin{itemize}
    \item MODEL: GENERAL
    \item A: Amplitude[V]
    \item F: Frequenz[Hz]
\end{itemize}

\subsection*{Rechtecksignal}
\begin{itemize}
    \item MODEL: SQUARE
    \item VONE: "HIGH"
    \item VZERO: "LOW"
    \item Px: $\rightarrow$ Abb. \ref{fig:Konfiguration Rechtecksignal}
    \item Überprüfen: Plot
\end{itemize}

\begin{figure}[h]
    \centering
    % \includegraphics[width=1\linewidth]{Microcap-Bilder/Rechtecksignal.PNG}
    \caption{Konfiguration Rechtecksignal}
    \label{fig:Konfiguration Rechtecksignal}
\end{figure}

\newpage

\subsection*{OPV}

\subsubsection*{Schmitt-Trigger}
Für einen Schmitt-Trigger mit unsymmetrischer Versorgung (0V, 5V), der ungefähr zwischen 0V und 4,5V hin-und herschaltet (vorausgesetzt er wurde richtig beschalten ;P), müssen folgende Einstellungen verwendet werden. (siehe Abb.)

\begin{itemize}
    \item MODEL: \$GENERIC
    \item VNS: 0.6
    \item VSS: 5
    \item VEE: 0
    \item VPS: 4
\end{itemize}

\begin{figure}[h]
    \centering
    % \includegraphics[width=0.65\linewidth]{Microcap-Bilder/invertierenderSchmitttrigger.PNG}
    \caption{Einstellungen für Schmitt-trigger}
    \label{fig:Einstellungen für Schmitt-trigger}
\end{figure}

\subsubsection*{Power Supplies}
Platziert man einen OPV, werden manchmal von Microcap automatisch Labels hinzugefügt, die den OPV mit 15V und -15V versorgen.\\
Diese Labels kann man löschen und durch eine andere Versorgung ersetzen (siehe Fixed Analog) oder man verändert die Werte unter Power Supplies:

\begin{figure}[h]
    \centering
    % \includegraphics[width=1\linewidth]{Microcap-Bilder/PowerSupplies.PNG}
    \caption{Power Supplies}
    \label{fig:Power Supplies}
\end{figure}

\subsubsection*{Transformator}

\begin{itemize}
    \item VALUE: Primärspule, Sekundärspule, Kopplungsfaktor\\
    z.Bsp: VALUE = 120u, 1m, 0.1
    \item der Pin "Plus/Minus output" ist die Primärspule und der Pin "Plus/Minus Input" ist die Sekundärspule.
\end{itemize}

\subsection{Fixed Analog - Spannungsversorgung}
Dieses Bauteil liefert eine fixe Gleichspannung.\\
Dadurch kann man sich mehrere Verbindungen, die beispielsweise bei einer Spannungsquelle nötig wären, ersparen.\\
(Findet man indem man danach sucht (siehe \textit{Komponenten suchen}))

\newpage

\subsection{Simulationspunkte}
$\rightarrow$ Text Mode $\rightarrow$ Text platzieren und passenden Namen geben (z. Bsp: U...) $\rightarrow$ mit Leitung verbinden (erkennt man am roten Punkt)

\begin{figure}[h]
    \centering
    % \includegraphics[width=0.5\linewidth]{Textmode.PNG}
    \caption{Text Mode}
    \label{fig:Text Mode}
\end{figure}

\newpage

\subsection{Simulation}

\subsubsection*{Zeitbereich}
\textbf{Analysis $\rightarrow$ Transient}

\begin{itemize}
    \item Maximum Run Time: bis zu diesem Zeitpunkt wird simuliert
    \item Output Start Time (tstart): Zeitpunkt, an der die Simulation beginnt
    \item Maximum Time Step: $\rightarrow$ 0
    \item Number of Points: Bestimmt die Anzahl der simulierten Punkte $\rightarrow$ je höher, desto länger dauert die Simulation
    \item Temperature: Temperatur
\end{itemize}

\begin{figure}[h]
    \centering
    % \includegraphics[width=1\linewidth]{Microcap-Bilder/Transientenanalyse-Diagramme.PNG}
    \caption{Transientenanalyse - Einstellungen der Graphen und Achsen}
    \label{fig:Transientenanalyse - Einstellungen der Graphen und Achsen}
\end{figure}

\textbf{Abb. \ref{fig:Transientenanalyse - Einstellungen der Graphen und Achsen} von links nach rechts:}
\begin{itemize}
    \item \begin{itemize}
        \item Grün: Graph wird angezeigt
        \item Gelb: Graph wird simuliert, aber nicht angezeigt
        \item Rot: Graph wird weder simuliert noch angezeigt
    \end{itemize}
    \item \begin{itemize}
        \item Grün: X-Achse linear skaliert
        \item Blau: X-Achse logarithmisch skaliert
    \end{itemize}
    \item \begin{itemize}
        \item Grün: Y-Achse linear skaliert
        \item Blau: Y-Achse logarithmisch skaliert
    \end{itemize}
    \item Farbenauswahl für die Graphen
    \item Page: Auf welcher Seite der Graph ist (1, 2, ...)
    \item P: In welchem Diagramm der Graph ist (1, 2, ...)
    \item X Expression: T $\rightarrow$ Zeit
    \item Y Expression: \begin{itemize}
        \item v(Simulationspunkt) - Spannung (z. Bsp. v(Uin))
        \item I(...) - Strom (z. Bsp. I(C1))
        \item normale Rechenoperationen können verwendet werden (z.Bsp. v(Uin)-v(Uout))
    \end{itemize}
    \item X-Range: Der Bereich der x-Achse, wobei Auto die einfachste Variante ist; TMAX ist die Maximalzeit. Ansonsten können hier auch - mit Beistrichen getrennt - der Endwert, Startwert und der Unterteilungsabstand festgelegt werden.
    \item Y-Range: gleiche Einstellung wie bei X-Range
\end{itemize}

\subsubsection*{Bodediagramm}

\begin{itemize}
    \item Frequency Range: letzter simulierter Frequenzpunkt, erster simulierter Frequenzpunkt (z. Bsp: 1meg, 10k)
    \item X-Achsen: logarithmisch skalieren
    \item X-Expression: F $\rightarrow$ Frequenz
    \item Y-Expression: \begin{itemize}
        \item Betrag: z. Bsp. dB(v(Uout))
        \item Phase: z. Bsp. ph(v(Uout))
    \end{itemize}
\end{itemize}

\begin{figure}[h]
    \centering
    % \includegraphics[width=1\linewidth]{Microcap-Bilder/ACAnalyse-Limits.PNG}
    \caption{AC-Analyse $\rightarrow$ Bodediagramm}
    \label{fig:Bodediagramm}
\end{figure}
% Folgende Schritte müssen befolgt werden, um in \textbf{Microcap} simulieren zu können:
% \begin{enumerate}
%     % \item Zunächst muss eine Datei erstellt werden: \\
%     % \verb|File| $\rightarrow$ \verb|New| $\rightarrow$ \verb|Schematic File (.cir)| $\rightarrow$ \verb|OK|

%     \item \underline{Komponentenauswahl} \\
%     \begin{itemize}
%         \item Allgemeine Bauteile (Widerstände, Kondensatoren, Quellen, etc.): \verb|Component| $\rightarrow$ \verb|Analog Primitives| $\rightarrow$ \verb|Benötigtes Bauteil|
%         \item Spezifische Bauteile (Transistoren, OPVs, etc.): \verb|Component| $\rightarrow$ \verb|Find Component| $\rightarrow$ \verb|Eingabe des Bauteilnamens| $\rightarrow$ \verb|Auswählen des Bauteils| $\rightarrow$ \verb|OK|
%     \end{itemize}
%     \item \underline{Verbinden der Bauteile} \\
%     Anklicken von Wire Mode. \todo{Add pic}

%     \item \underline{Konfiguration der Bauteile} \\
%     \textbf{Beispiel}
%     Pulse Source

%     \item \underline{Platzierung von Simulationspunkten} \\
%     Anklicken von \verb|Text Mode| und gewünschten Namen eingeben, welcher in der Simulation ausgewählt werden kann. Danach auf die gewünschte Stelle in der Schaltung ziehen, an der dann ein roter Punkt erscheinen sollte.

%     \item \underline{Einstellen der Simulation} \\
%     \begin{itemize}
%         \item \textbf{Im Zeitbereich} \\
%         \verb|Analysis| $\rightarrow$ \verb|Transient|
%         \begin{itemize}
%             \item \underline{Maximum Run Time}: Gesamtzeit der Simulation.
%             \item \underline{Number of Points}: Die Anzahl der simulierten Punkte.
%             \item \underline{Page}: Die Seite auf der der simulierte Graph gezeigt wird.
%             \item \underline{P}: Jede Seite kann mehrere Plots enthalten ($\approx$Graph); ist dies der Fall, werden sie untereinander dargestellt.
%             \item \underline{X Expression}: Wird im Normalfall mit $t$ als Zeit beschrieben, kann aber jedoch für etwas anderes verwendet werden.
%             \item \underline{Y Expression}: Dort sollten die jeweiligen Simulationspunkte ausgewählt werden, z.B.Spannungsverhältnisse, deren dB-Werte oder andere Funktionen bzw. Funktionswerte.
%             \item \underline{X Range}: Der Bereich der x-Achse, wobei \verb|Auto| die einfachste Variante ist; \verb|TMAX| ist die Maximalzeit.\\
%             Ansonsten können hier auch - mit Beistrichen getrennt - der Endwert, Startwert und der Unterteilungsabstand festgelegt werden.
%             \item \underline{Y Range}: Der Bereich der y-Achse; Optionen sind hier die selben wie bei \verb|X Range|.
%             \todo{Verschiedene Darstellungsmethoden hinzufügen.}
%         \end{itemize}

%     \end{itemize}
% \end{enumerate}
% \todo{Wirklich? Wie man eine Datei macht??}

\end{document}