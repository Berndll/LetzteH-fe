\chapter{OPV-Schaltungen}

\section{Verstärker}

\subsection{Leerlaufverstärkung}
Merke, dass kein Strom in den OPV fließt; es gilt:
\begin{align}
    V = \frac{U_a}{U_d}
\end{align}
\begin{itemize}
    \item $V$ ist die Verstärkung
    \item $U_a$ ist die Ausgangsspannung
    \item $U_d$ ist die Differenzspannung (zwischen dem "+"- und "-"-Eingang).
\end{itemize}
\begin{center}
\begin{circuitikz}
    \coordinate (opAmp) at (0,0);
    \coordinate (opIn+) at ($(opAmp) + (-1.25,.5)$);
    \coordinate (opIn+Label) at ($(opIn+) + (-0.25,0)$);
    \coordinate (opIn-) at ($(opAmp) + (-1.25,-.5)$);
    \coordinate (opIn-Label) at ($(opIn-) + (-0.25,0)$);
    \coordinate (opOut) at ($(opAmp) + (1.25,0)$);

    \draw(opAmp) node[op amp, yscale=-1] {};
        
    \draw[->, thick, blue] (opIn+Label) -- (opIn-Label) node[above left, blue] {$U_D$};
    \draw[->, thick, blue] (opOut) -- +(0,-1) node[above right, blue] {$U_a$};
\end{circuitikz}
\end{center}
\subsection{Impedanzwandler}
Ist dazu da, um dem folgenden System mehr Strom liefern zu können.\\
Beachte, dass die Verstärkung hier $V=0$ ist weil $U_a=U_e$.
\begin{center}
\begin{circuitikz}
    \coordinate (opAmp) at (3,-0.5);
    \coordinate (opIn+) at ($(opAmp) + (-1.25,.5)$);
    \coordinate (opIn+Label) at ($(opIn+) + (-0.25,0)$);
    \coordinate (opIn-) at ($(opAmp) + (-1.25,-.5)$);
    \coordinate (opIn-Label) at ($(opIn-) + (-0.25,0)$);
    \coordinate (opOut) at ($(opAmp) + (1.25,0)$);

    \draw(opAmp) node[op amp, yscale=-1] {};

    \draw[black] (0,0) -- (opIn+);
    \draw[black] (opOut) -- +(1,0);
    \draw[black] (opOut) -- +(0,-1.5) -- +(-2.5,-1.5) -- (opIn-);

    \draw[->, thick, blue] (opIn+Label) -- (opIn-Label) node[above left, blue]    {$U_d$};
    \draw[->, thick, blue] ($(opOut) + (1.25,0)$) -- +(0,-2) node[above right, blue] {$U_a$};
    \draw[->, thick, blue] (-0.25,0) -- +(0,-2.5) node[above right, blue] {$U_e$};

    \filldraw[black]
        (opIn+) circle (1.5pt)
        (opIn-) circle (1.5pt)
        (opOut) circle (1.5pt);
\end{circuitikz}
\end{center}

\subsection{Invertierender Verstärker}

\begin{center}
\begin{circuitikz}
        \coordinate (opAmp) at (4,-0.5);
        \coordinate (opIn+) at ($(opAmp) + (-1.25,.5)$);
        \coordinate (opIn+Label) at ($(opIn+) + (-0.25,0)$);
        \coordinate (opIn-) at ($(opAmp) + (-1.25,-.5)$);
        \coordinate (opIn-Label) at ($(opIn-) + (-0.25,0)$);
        \coordinate (opOut) at ($(opAmp) + (1.25,0)$);
    
        % OpAmp
        \draw(opAmp) node[op amp] {};
        
        % Resistors
        \draw(0,0) to[R, l=$R_1$] (opIn+);
        \draw($(opIn+) + (0,1.5)$) to[R, l=$R_2$] ($(opIn+) + (2.5,1.5)$);
 
        % Wires
        \draw[black] (opOut) -- +(1,0);
        \draw[black] (opIn+) -- ($(opIn+) + (0,1.5)$);
        \draw[black] ($(opIn+) + (2.5,1.5)$) -- (opOut);

        \draw[black] (opIn-) -- +(0,-1);
        
        % Ground
        \draw[ultra thick, black] ($(opIn-) +(-0.25,-1)$) -- +(0.5,0);

        % Voltages
        \draw[->, thick, blue] (opIn+Label) -- (opIn-Label) node[above left, blue] {$U_d$};
\end{circuitikz}
\end{center}

$R_e$ ist der Eingangswiderstand der Schaltung.\\

Hier gilt: $R_e=R_1$

\begin{align}
    V = \frac{U_a}{U_d}=-\frac{R_2}{R_1}
\end{align}

\subsection{Nicht-Invertierender Verstärker}
\begin{center}
\begin{circuitikz}
        \coordinate (opAmp) at (2,-0.5);
        \coordinate (opIn+) at ($(opAmp) + (-1.25,.5)$);
        \coordinate (opIn+Label) at ($(opIn+) + (-0.25,0)$);
        \coordinate (opIn-) at ($(opAmp) + (-1.25,-.5)$);
        \coordinate (opIn-Label) at ($(opIn-) + (-0.25,0)$);
        \coordinate (opOut) at ($(opAmp) + (1.25,0)$);
    
        % OpAmp
        \draw(opAmp) node[op amp, yscale=-1] {};

        % Resistors
        \draw(opOut) to[R, l=$R_1$] ($(opOut) + (0,-2)$);
        \draw($(opOut) + (0,-2)$) to[R, l=$R_2$] ($(opOut) + (0,-4)$);

        % Wires
        \draw[black] (opOut) -- +(1,0);
        \draw[black] (opIn+) -- ($(opIn+) + (-1,0)$);
        \draw[black] (opIn-) -- ($(opIn+) + (0,-2.5)$) -- ($(opOut) + (0,-2)$);

        % Voltages
        \draw[->, thick, blue] (opIn+Label) -- (opIn-Label) node[above left, blue] {$U_d$};
\end{circuitikz}
\end{center}

$R_e$ ist der Eingangswiderstand der Schaltung.\\

Hier gilt: $R_e=\infty$

\begin{align}
    V = \frac{U_a}{U_d}=\frac{R_1+R_2}{R_2}
\end{align}

\section{Schmitttrigger}

\subsection{Invertierender Schmitttrigger}
\begin{align}
    \frac{U_a}{U_e}=\frac{R_1+R_2}{R_2}
\end{align}

\begin{center}
\begin{circuitikz}
    \coordinate (opAmp) at (2,-0.5);
    \coordinate (opIn+) at ($(opAmp) + (-1.25,.5)$);
    \coordinate (opIn+Label) at ($(opIn+) + (-0.25,0)$);
    \coordinate (opIn-) at ($(opAmp) + (-1.25,-.5)$);
    \coordinate (opIn-Label) at ($(opIn-) + (-0.25,0)$);
    \coordinate (opOut) at ($(opAmp) + (1.25,0)$);

    % OpAmp
    \draw(opAmp) node[op amp] {};

    % Resistors
    \draw(opOut) to[R, l=$R_1$] ($(opOut) + (0,-2)$);
    \draw($(opOut) + (0,-2)$) to[R, l=$R_2$] ($(opOut) + (0,-4)$);

    % Wires
    \draw[black] (opOut) -- +(1,0);
    \draw[black] (opIn+) -- ($(opIn+) + (-1,0)$);
    \draw[black] (opIn-) -- ($(opIn+) + (0,-2.5)$) -- ($(opOut) + (0,-2)$);

    % Voltages
    \draw[->, thick, blue] (opIn+Label) -- (opIn-Label) node[above left, blue] {$U_d$};
\end{circuitikz}
\end{center}

\subsection{Nicht-Invertierender Schmitttrigger}
\begin{align}
    \frac{U_a}{U_e}=\frac{R_2}{R_1}
\end{align}

\begin{center}
\begin{circuitikz}
        \coordinate (opAmp) at (4,-0.5);
        \coordinate (opIn+) at ($(opAmp) + (-1.25,.5)$);
        \coordinate (opIn+Label) at ($(opIn+) + (-0.25,0)$);
        \coordinate (opIn-) at ($(opAmp) + (-1.25,-.5)$);
        \coordinate (opIn-Label) at ($(opIn-) + (-0.25,0)$);
        \coordinate (opOut) at ($(opAmp) + (1.25,0)$);
    
        % OpAmp
        \draw(opAmp) node[op amp, yscale=-1] {};
        
        % Resistors
        \draw(0,0) to[R, l=$R_1$] (opIn+);
        \draw($(opIn+) + (0,1.5)$) to[R, l=$R_2$] ($(opIn+) + (2.5,1.5)$);
    
        % Wires
        \draw[black] (opOut) -- +(1,0);
        \draw[black] (opIn+) -- ($(opIn+) + (0,1.5)$);
        \draw[black] ($(opIn+) + (2.5,1.5)$) -- (opOut);

        \draw[black] (opIn-) -- +(0,-1);
        
        % Ground
        \draw[ultra thick, black] ($(opIn-) +(-0.25,-1)$) -- +(0.5,0);

        % Voltages
        \draw[->, thick, blue] (opIn+Label) -- (opIn-Label) node[above left, blue] {$U_d$};
\end{circuitikz}
\end{center}

\newpage

\section{Addierer}
Wenn $R_1=R_2=R_3$ dann gilt $U_a=-(U_{R_1}+U_{R_2})$.

\textbf{Berechnung mit Teilströmen}
\begin{align}
    I_1&=\frac{U_{e_1}}{R_1} \\
    I_2&=\frac{U_{e_2}}{R_2}
\end{align}
\begin{align}   
    U_{R_g}&=R_g\cdot(I_1+I_3) \\
    U_{R_g}&=R_g\cdot(\frac{U_{e_1}}{R_1}+\frac{U_{e_2}}{R_2}) \\
    U_{R_g}&=R_g\cdot(\frac{U_{e_1}\cdot R_g}{R_1}+\frac{U_{e_2}\cdot R_g}{R_2})
\end{align}
\todo{Woher das Minus auf einmal?}

\textbf{Berechnung mit Überlagerungsprinzip} \\

\underline{$U_{e_1}$ wirkt, $U_{e_2}=0$:} \hspace{2cm} $U_a'=\frac{R_g}{R_1}\cdot U_{e_1}$ \\

\underline{$U_{e_2}$ wirkt, $U_{e_2}=1$:} \hspace{2cm} $U_a''=\frac{R_g}{R_2}\cdot U_{e_2}$ \\

\underline{Gesamt:}
\begin{align}  
    U_a&=U_a'+U_a''  \\
    U_a&=\frac{R_g}{R_1}\cdot U_{e_1}+\frac{R_g}{R_2}\cdot U_{e_2} \\
    U_a&=-(U_{e_1}\cdot\frac{R_g}{R_1}+U_{e_2}\cdot\frac{R_g}{R_2})
\end{align}

\section{Subtrahierer}
\subsection{Typ 1}
\begin{align}
    U_a=(U_{e_2}-U_{e_1}\cdot{\frac{R_2}{R_1}})
\end{align}

\subsection{Typ 2}
\begin{align}
    U_a=(\sum B - \sum A)\cdot\frac{R_2}{R_1}
\end{align}

\subsection{Typ 3}
\underline{Berechnung mit Überlagerungsprinzip} \\
\underline{$U_{e_1}$ wirkt, $U_{e_2}=0$:}
\begin{align}
    U_a'=\frac{R_4+R_2}{R_2}\cdot U_{e_1}\cdot\frac{R_3}{R_1+R_3}
\end{align}

\underline{$U_{e_2}$ wirkt, $U_{e_2}=1$:}
\begin{align}
    U_a''=-U_{e_2}\cdot \frac{R_4}{R_2}
\end{align}

\underline{Gesamt}
\begin{align}
    U_a&=U_a'+U_a'' \\
    U_a&=U_{e_1}\cdot (\frac{R_4+R_2}{R_2}\cdot\frac{R_3}{R_1+R_3})-U_{e_2}\cdot\frac{R_4}{R_2}
\end{align}

Wenn alle $R$ gleich groß sind, gilt: $U_a=U_{e_1}-U_{e_2}$

\todo{Berechnung mit Teilspannungen}

\section{Integrator \and Differentiator}


\section{Pegelwandler}
Hier gilt:
\begin{align}
    V=-\frac{R_2}{R_1}=\frac{\Delta U_a}{\Delta U_e}
\end{align}

\textbf{Beispiel}\\
Das Eingangssignal von $U_e=-1V$ bis $+1V$ soll am Ausgang zu $U_a=0V$ bis $+5V$ gewandelt werden.
\begin{align}
    &V=-\frac{R_2}{R_1}=\frac{\Delta U_a}{\Delta U_e}   \\
    &V=-\frac{5V-0V}{1V-(-1V)}=\frac{5V}{2V}            \\
    &V=-\frac{5k\Omega}{2k\Omega}
\end{align}

\begin{align}
    \frac{U_{R_2}}{U_e-U_a}&=\frac{R_2}{R_1+R_2}                \\
    U_{R_2}&=(U_e-U_a)-\frac{R_2}{R_1+R_2}                      \\
    U_{R_2}&=(1V-0V)-\frac{5k\Omega}{2k\Omega+5k\Omega}         \\
    U_{R_2}&\approx 0,715V                                      \\
    \Rightarrow U_+=U_a+U_{R_2}&=0V+0,714V=0,714V
\end{align}

\section{Instrumentation-Amplifier}