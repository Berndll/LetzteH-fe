\chapter{OPV-Schaltungen}

Prinzipiell gilt beim OPV immer: $V = \frac{U_a}{U_D}$
\begin{itemize}
    \item $V$ ist die Verstärkung
    \item $U_a$ ist die Ausgangsspannung
    \item $U_D$ ist die Differenzspannung (zwischen dem "+"- und "-"-Eingang).
\end{itemize}

Wichtig zu wissen ist, dass \textbf{kein Strom in den OPV fließt}. \\
Außerdem versucht er die Differenzspannung \textbf{$U_D$ immer auf $0V$} zu regeln!

\begin{center}
\begin{circuitikz}
    \coordinate (opAmp) at (0,0);
    \coordinate (opIn+) at ($(opAmp) + (-1.25,.5)$);
    \coordinate (opIn+Label) at ($(opIn+) + (-0.25,-0.1)$);
    \coordinate (opIn-) at ($(opAmp) + (-1.25,-.5)$);
    \coordinate (opIn-Label) at ($(opIn-) + (-0.25,0.1)$);
    \coordinate (opOut) at ($(opAmp) + (1.25,-0.1)$);

    \draw(opAmp) node[op amp, yscale=-1] {};
    \draw (1.25,-1.15) node[rground]{};
        
    \draw[->, thick, blue] (opIn+Label) -- (opIn-Label) node[above left, blue] {$U_D$};
    \draw[->, thick, blue] (opOut) -- +(0,-0.9) node[above right, blue] {$U_a$};

    \draw (-1,0.49) -- (-1.45,0.49);
    \draw (-1,-0.49) -- (-1.45,-0.49);

    \draw (1.25,0) circle (1.5pt);
	\draw (1.25,-1.1) circle (1.5pt); 
    \draw (-1.5,0.49) circle (1.5pt);
	\draw (-1.5,-0.49) circle (1.5pt); 
\end{circuitikz}
\end{center}

Aufgrund der typischen Verstärkung von $20 ... 200.000$, erreicht der OPV seine Aussteuergrenze, weswegen gilt:
\begin{itemize}
    \item \underline{$U_D > 0V \Rightarrow U_a = V_{CC}$} (positive Versorgungsspannung) und
    \item \underline{$U_D < 0V \Rightarrow U_a = V_{SS}$} (negative Versorgungsspannung). 
\end{itemize}
Die verschiedenen Schaltungen verwenden dieses Verhalten zu deren Gunsten. \\

Bei \textbf{symmetrischer Versorgung} gilt: $V_{SS} = -V_{CC}$. \\
Bei \textbf{unsymmetrischer Versorgung} gilt: $V_{SS} \neq -V_{CC}$ (typ. $GND$). \\

\newpage

\section{Verstärker}

\subsection{Impedanzwandler}
Der Impedanzwandler ist dazu da, um dem folgenden System mehr Strom liefern zu können; weil $U_a = U_e$ gilt, ist die Verstärkung hier $V=1$:
\begin{center}
\begin{circuitikz}
    \coordinate (opAmp) at (3,-0.5);
    \coordinate (opIn+) at ($(opAmp) + (-1.25,.5)$);
    \coordinate (opIn+Label) at ($(opIn+) + (-0,-0.1)$);
    \coordinate (opIn-) at ($(opAmp) + (-1.25,-.5)$);
    \coordinate (opIn-Label) at ($(opIn-) + (0,0.1)$);
    \coordinate (opOut) at ($(opAmp) + (1.25,0)$);

    \draw(opAmp) node[op amp, yscale=-1] {};
    \draw (0,-2.05) node[rground]{};
    \draw (5.5,-2.05) node[rground]{};

    \draw[black] (0.05,0) -- (opIn+);
    \draw[black] (opOut) -- +(1.2,0);
    \draw[black] (opOut) -- +(0,-1.5) -- +(-2.5,-1.5) -- (opIn-);

    \draw[->, thick, blue] (opIn+Label) -- (opIn-Label) node[above left, blue]    {$U_D$};
    \draw[->, thick, blue] ($(opOut) + (1.25,-0.1)$) -- +(0,-1.3) node[above right, blue] {$U_a$};
    \draw[->, thick, blue] (0,-0.1) -- +(0,-1.8) node[above right, blue] {$U_e$};


    \draw (0,0) circle (1.5pt);
	\draw (0,-2) circle (1.5pt); 
    \draw (5.5,-2) circle (1.5pt);
	\draw (5.5,-0.49) circle (1.5pt); 

    \filldraw[black]
        (opIn+) circle (1.5pt)
        (opIn-) circle (1.5pt)
        (opOut) circle (1.5pt);
\end{circuitikz}
\end{center}
Dieser Wandler wird auch "Spannungsfolger" genannt.

\subsection{Nicht-Invertierender Verstärker}
Diese Schaltung verstärkt $U_e$ bis zur Aussteuergrenze; z.B.: bei $V = +2$ wird $+2,5V$ zu $+5V$

\begin{center}
\begin{circuitikz}
    \node[op amp, noinv input up] at (0,0) (opamp) {};
    \draw(1.5,0) to[R, l=$R_1$] (1.5,-2);
    \draw(1.5,-2) to[R, l=$R_2$] (1.5,-4);
    \draw (-4.55,-4) node[rground]{};
    \draw (1.5,-4) node[rground]{};
    \draw (3.55,-4) node[rground]{};

    \draw (-2,-0.49) -- (-1,-0.49);
    \draw (-2,-0.49) -- (-2,-2);
    \draw (-2,-2) -- (1.5,-2);

    
    \draw (-1,0.49) -- (-4.5,0.49);
    \draw (1,0) -- (3.5,0);

    \draw (-4.55,0.49) circle (1.5pt);
    \draw (-4.55,-3.95) circle (1.5pt); 
    \draw (3.55,0) circle (1.5pt);
    \draw (3.55,-3.95) circle (1.5pt); 
    \draw[black,fill=black] (1.5,-2) circle (1.5pt);
    \draw[black,fill=black] (1.5,0) circle (1.5pt);

    \draw[->, thick, blue] (-1.2,0.4) -- (-1.2,-0.4) node[above left, blue] {$U_D$};
    \draw[->, thick, blue] (-4.55,0.4) -- (-4.55,-3.8) node[above left, blue] {$U_e$};
    \draw[->, thick, blue] (3.55,-0.1) -- (3.55,-3.8) node[above left, blue] {$U_a$};
\end{circuitikz}
\end{center}

Für die Dimensionierung der Widerstände gilt:
\begin{align}
    V = \frac{U_a}{U_e}=\frac{R_1+R_2}{R_2}
\end{align}
Es wird einer der Widerstände angenommen (typischerweise $R_2$) und berechnet den zweiten.

\subsection{Invertierender Verstärker}
Diese Schaltung invertiert $U_e$ und verstärkt das Signal bis zur Aussteuergrenze; z.B.: bei $V = -2$ wird $+2,5V$ zu $-5V$

\begin{center}
\begin{circuitikz}
    \draw node [op amp] (A1) {};
    \draw(-4,0.49) to[R, l=$R_1$] (-2,0.49);
    \draw(-1,2) to[R, l=$R_2$] (1,2);
    \draw (-4.55,-1.55) node[rground]{};
    \draw (-2,-1.5) node[rground]{};
    \draw (2.55,-1.55) node[rground]{};

    \draw (-4,0.49) -- (-4.5,0.49);
    \draw (-2,-0.49) -- (-1,-0.49);
    \draw (-2,-0.49) -- (-2,-1.5);
    \draw (-2,0.49) -- (-1,0.49);
    \draw (-1.5,0.49) -- (-1.5,2);
    \draw (-1.5,2) -- (-1,2);
    \draw (1,2) -- (1.5,2);
    \draw (1.5,2) -- (1.5,0);
    \draw (1,0) -- (2.5,0);

    \draw (-4.55,0.49) circle (1.5pt);
    \draw (-4.55,-1.5) circle (1.5pt); 
    \draw (2.55,0) circle (1.5pt);
    \draw (2.55,-1.5) circle (1.5pt); 
    \draw[black,fill=black] (-1.5,0.49) circle (1.5pt);
    \draw[black,fill=black] (1.5,0) circle (1.5pt);

    \draw[->, thick, blue] (-1.2,0.4) -- (-1.2,-0.4) node[above left, blue] {$U_D$};
    \draw[->, thick, blue] (-4.55,0.4) -- (-4.55,-1.4) node[above left, blue] {$U_e$};
    \draw[->, thick, blue] (2.55,-0.1) -- (2.55,-1.4) node[above left, blue] {$U_a$};
    \draw[<-, thick, blue] (-1,2.85) -- (1,2.85) node[above, midway, blue] {$U_{2}$};
\end{circuitikz}
\end{center}

Für die Dimensionierung der Widerstände gilt:
\begin{align}
    V = \frac{U_a}{U_e}=-\frac{R_2}{R_1}
\end{align}
Es wird einer der beiden angenommen und der andere berechnet.

\newpage

\section{Schmitttrigger}
Ein Schmitttrigger ist eine Schaltung, bei dem eine gewisse Schaltschwelle unter- bzw. überschritten werden muss, um seinen Zustand ($V_{CC}$ oder $V_{SS}$) zu ändern.

\subsection{Nicht-Invertierender Schmitttrigger}
Der nicht-invertierende Schmitttrigger gibt \underline{$V_{CC}$} bei überschreiten der oberen Schaltschwelle aus und \underline{$V_{SS}$} bei unterschreiten der unteren Schaltschwelle.

\todo{Fix schematic}

\begin{center}
\begin{circuitikz}
    \draw node [op amp,  noinv input up](A1){};
    \draw(-4,0.49) to[R, l=$R_1$] (-2,0.49);
    \draw(-1,2) to[R, l=$R_2$] (1,2);
    \draw (-4.55,-1.55) node[rground]{};
    \draw (-2,-1.5) node[rground]{};
    \draw (2.55,-1.55) node[rground]{};

    \draw (-4,0.49) -- (-4.5,0.49);
    \draw (-2,-0.49) -- (-1,-0.49);
    \draw (-2,-0.49) -- (-2,-1.5);
    \draw (-2,0.49) -- (-1,0.49);
    \draw (-1.5,0.49) -- (-1.5,2);
    \draw (-1.5,2) -- (-1,2);
    \draw (1,2) -- (1.5,2);
    \draw (1.5,2) -- (1.5,0);
    \draw (1,0) -- (2.5,0);

    \draw (-4.55,0.49) circle (1.5pt);
    \draw (-4.55,-1.5) circle (1.5pt); 
    \draw (2.55,0) circle (1.5pt);
    \draw (2.55,-1.5) circle (1.5pt); 
    \draw[black,fill=black] (-1.5,0.49) circle (1.5pt);
    \draw[black,fill=black] (1.5,0) circle (1.5pt);

    \draw[->, thick, blue] (-1.2,0.4) -- (-1.2,-0.4) node[above left, blue] {$U_D$};
    \draw[->, thick, blue] (-4.55,0.4) -- (-4.55,-1.4) node[above left, blue] {$U_e$};
    \draw[->, thick, blue] (2.55,-0.1) -- (2.55,-1.4) node[above left, blue] {$U_a$};
\end{circuitikz}
\end{center}

Zur Dimensionierung der Widerstände gilt:
\begin{align}
    \frac{\Delta U_a}{\Delta U_e}=\frac{R_2}{R_1}
\end{align}

Die Hysterese des nicht-invertierendes Schmitttriggers sieht folgendermaßen aus:
\begin{center}
\begin{circuitikz} [scale=0.75]
    \draw[<->] (-3,0) -- (3,0) node[right, above] {$U_e$};
    \draw[<->] (0,-3) -- (0,3) node[right, above] {$U_a$};

    \draw[->, ultra thick, Green] (2,2) -- +(2,0);
    \draw[->, ultra thick, Green] (-2,-2) -- +(4,0);
    \draw[->, ultra thick, Green] (2,-2) -- +(0,4);

    \draw[->, ultra thick, red] (2,2) -- +(-4,0);
    \draw[->, ultra thick, red] (-2,-2) -- +(-2,0);
    \draw[->, ultra thick, red] (-2,2) -- +(0,-4);
\end{circuitikz}
\end{center}

\subsubsection*{Beispiel}
Ein Schmitttrigger soll $2V$ bis $3V$ am Eingang, zu $0V$ bis $5V$ am Ausgang umsetzen.

\begin{align}
    \frac{\Delta U_a}{\Delta U_e} &= \frac{5V - 0V}{3V - 2V} = \frac{5V}{1V}
\end{align}

Annahme: $R_1 = 10k\Omega$
\begin{align}
    \frac{\Delta U_a}{\Delta U_e} &= \frac{R_2}{R_1} \\
    \Rightarrow R_2 &= R_1 \cdot \frac{U_a}{U_e} \\
    R_2 &= 10k\Omega \cdot \frac{5V}{1V} = 50k\Omega
\end{align}

Außerdem muss der U-Minus-Eingang um die Mitte der Schaltschwellen hinaufgeschoben werden, also hier: $2V$ und $3V$ wird zu einer Mittensspannung von $2,5V$. Dies kann einfach durch einen Spannungsteiler erreicht werden.


\begin{center}
\begin{circuitikz}
    \draw node [op amp,  noinv input up](A1){};
    \draw(-4,0.49) to[R, l=$R_1$] (-2,0.49);
    \draw(-1,2) to[R, l=$R_2$] (1,2);
    \draw (-4.55,-1.55) node[rground]{};
    \draw (-2,-1.5) node[rground]{};
    \draw (2.55,-1.55) node[rground]{};

    \draw (-4,0.49) -- (-4.5,0.49);
    \draw (-2,-0.49) -- (-1,-0.49);
    \draw (-2,-0.49) -- (-2,-1.5);
    \draw (-2,0.49) -- (-1,0.49);
    \draw (-1.5,0.49) -- (-1.5,2);
    \draw (-1.5,2) -- (-1,2);
    \draw (1,2) -- (1.5,2);
    \draw (1.5,2) -- (1.5,0);
    \draw (1,0) -- (2.5,0);

    \draw (-4.55,0.49) circle (1.5pt);
    \draw (-4.55,-1.5) circle (1.5pt); 
    \draw (2.55,0) circle (1.5pt);
    \draw (2.55,-1.5) circle (1.5pt); 
    \draw[black,fill=black] (-1.5,0.49) circle (1.5pt);
    \draw[black,fill=black] (1.5,0) circle (1.5pt);

    \draw[->, thick, blue] (-1.2,0.4) -- (-1.2,-0.4) node[above left, blue] {$U_D$};
    \draw[->, thick, blue] (-4.55,0.4) -- (-4.55,-1.4) node[above left, blue] {$U_e$};
    \draw[->, thick, blue] (2.55,-0.1) -- (2.55,-1.4) node[above left, blue] {$U_a$};
\end{circuitikz}
\end{center}

\newpage

\subsection{Invertierender Schmitttrigger}
Der invertierende Schmitttrigger gibt \underline{$V_{SS}$} bei überschreiten der oberen Schaltschwelle aus und \underline{$V_{CC}$} bei unterschreiten der unteren Schaltschwelle.

\begin{center}
\begin{circuitikz}
    \node[op amp] at (0,0) (opamp) {};
    \draw(1.5,0) to[R, l=$R_1$] (1.5,-2);
    \draw(1.5,-2) to[R, l=$R_2$] (1.5,-4);
    \draw (-4.55,-4) node[rground]{};
    \draw (1.5,-4) node[rground]{};
    \draw (3.55,-4) node[rground]{};

    \draw (-2,-0.49) -- (-1,-0.49);
    \draw (-2,-0.49) -- (-2,-2);
    \draw (-2,-2) -- (1.5,-2);

    
    \draw (-1,0.49) -- (-4.5,0.49);
    \draw (1,0) -- (3.5,0);

    \draw (-4.55,0.49) circle (1.5pt);
    \draw (-4.55,-3.95) circle (1.5pt); 
    \draw (3.55,0) circle (1.5pt);
    \draw (3.55,-3.95) circle (1.5pt); 
    \draw[black,fill=black] (1.5,-2) circle (1.5pt);
    \draw[black,fill=black] (1.5,0) circle (1.5pt);

    \draw[->, thick, blue] (-1.2,0.4) -- (-1.2,-0.4) node[above left, blue] {$U_D$};
    \draw[->, thick, blue] (-4.55,0.4) -- (-4.55,-3.8) node[above left, blue] {$U_e$};
    \draw[->, thick, blue] (3.55,-0.1) -- (3.55,-3.8) node[above left, blue] {$U_a$};
\end{circuitikz}
\end{center}

Zur Dimensionierung der Widerstände gilt:
\begin{align}
    \frac{U_a}{U_e}=\frac{R_1+R_2}{R_2}
\end{align}


Die Hysterese des nicht-invertierendes Schmitttriggers sieht folgendermaßen aus:
\begin{center}
\begin{circuitikz} [scale=0.75]
    \draw[<->] (-3,0) -- (3,0) node[right, above] {$U_e$};
    \draw[<->] (0,-3) -- (0,3) node[right, above] {$U_a$};

    \draw[->, ultra thick, Green] (2,-2) -- +(2,0);
    \draw[->, ultra thick, Green] (-2,2) -- +(4,0);
    \draw[->, ultra thick, Green] (2,2) -- +(0,-4);

    \draw[->, ultra thick, red] (2,-2) -- +(-4,0);
    \draw[->, ultra thick, red] (-2,2) -- +(-2,0);
    \draw[->, ultra thick, red] (-2,-2) -- +(0,4);
\end{circuitikz}
\end{center}

\begin{align}
    U_{mitte}&=\frac{U_{Schwelle/Unten}+U_{Schwelle/Oben}}{2} \\
    R_2&=R_3\cdot\frac{U_{mitte}}{U_{Versorgung}-U_{mitte}} \\
    R_1&=\frac{R_2\cdot R_3}{R_2 + R_3}\cdot\frac{U_{Schwelle/Oben}-U_{Schwelle/Oben}-V_{CC}+GND}{U_{Schwelle/Unten}-U_{Schwelle/Oben}} 
\end{align}

\newpage

\section{Addierer}
Ein Addierer summiert die Eingangspannungen $U_{e_1}$ und $U_{e_2}$ und gibt das Ergebnis mit umgekehrtem Vorzeichen aus (wenn $R_1 = R_2=R_3$ dann gilt $U_a = -(U_{R_1}+U_{R_2})$).

\begin{center}
\begin{circuitikz}
        \draw node [op amp](A1){};
        \draw(-4,0.49) to[R, l=$R_1$] (-2,0.49);
        \draw(-4,2) to[R, l=$R_3$] (-2,2);
        \draw(-1,2) to[R, l=$R_2$] (1,2);
        \draw (-6.05,-1.55) node[rground]{};
        \draw (-4.55,-1.55) node[rground]{};
        \draw (-2,-1.5) node[rground]{};
        \draw (2.55,-1.55) node[rground]{};

        \draw (-2,2) -- (-1,2);
        \draw (-4,2) -- (-6,2);
        \draw (-4,0.49) -- (-4.5,0.49);
        \draw (-2,-0.49) -- (-1,-0.49);
        \draw (-2,-0.49) -- (-2,-1.5);
        \draw (-2,0.49) -- (-1,0.49);
        \draw (-1.5,0.49) -- (-1.5,2);
        \draw (-1.5,2) -- (-1,2);
        \draw (1,2) -- (1.5,2);
        \draw (1.5,2) -- (1.5,0);
        \draw (1,0) -- (2.5,0);

        \draw (-6.05,2) circle (1.5pt);
    	\draw (-6.05,-1.5) circle (1.5pt);     
        \draw (-4.55,0.49) circle (1.5pt);
    	\draw (-4.55,-1.5) circle (1.5pt); 
        \draw (2.55,0) circle (1.5pt);
    	\draw (2.55,-1.5) circle (1.5pt); 
        \draw[black,fill=black] (-1.5,0.49) circle (1.5pt);
    	\draw[black,fill=black] (1.5,0) circle (1.5pt);
        \draw[black,fill=black] (-1.5,2) circle (1.5pt);

        \draw[->, thick, blue] (-1.2,0.4) -- (-1.2,-0.4) node[above left, blue] {$U_D$};
        \draw[->, thick, blue] (-6.05,1.9) -- (-6.05,-1.4) node[above left, blue] {$U_{e_2}$};
        \draw[->, thick, blue] (-4.55,0.4) -- (-4.55,-1.4) node[above left, blue] {$U_{e_1}$};
        \draw[->, thick, blue] (2.55,-0.1) -- (2.55,-1.4) node[above left, blue] {$U_a$};
\end{circuitikz}
\end{center}

\subsubsection*{Berechnung mit Teilströmen}
\begin{align}
    I_1&=\frac{U_{e_1}}{R_1} \\
    I_2&=\frac{U_{e_2}}{R_2}
\end{align}
\begin{align}   
    U_{R_g}&=R_g\cdot(I_1+I_3) =R_g\cdot(\frac{U_{e_1}}{R_1}+\frac{U_{e_2}}{R_2}) \\
    U_{R_g}&=R_g\cdot(\frac{U_{e_1}\cdot R_g}{R_1}+\frac{U_{e_2}\cdot R_g}{R_2})
\end{align}

\subsubsection*{Berechnung mit Überlagerungsprinzip}

\underline{$U_{e_1}$ wirkt, $U_{e_2}=0$:} \hspace{2cm} $U_a'=\frac{R_g}{R_1}\cdot U_{e_1}$ \\

\underline{$U_{e_2}$ wirkt, $U_{e_2}=1$:} \hspace{2cm} $U_a''=\frac{R_g}{R_2}\cdot U_{e_2}$ \\

\underline{Gesamt:}
\begin{align}  
    U_a&=U_a'+U_a''  \\
    U_a&=\frac{R_g}{R_1}\cdot U_{e_1}+\frac{R_g}{R_2}\cdot U_{e_2} \\
    U_a&=-(U_{e_1}\cdot\frac{R_g}{R_1}+U_{e_2}\cdot\frac{R_g}{R_2})
\end{align}

\newpage

\section{Subtrahierer}
\subsection{Typ 1}
\begin{center}
\begin{circuitikz}
        \draw node [op amp](A1){};
        \draw(-4,0.49) to[R, l=$R_1$] (-2,0.49);
        \draw(-1,2) to[R, l=$R_2$] (1,2);
        \draw(-4,-0.49) to[R, l=$R_1$] (-2,-0.49);
        \draw(-2,-0.49) to[R, l=$R_2$] (-2,-2.49);
        \draw (-5.55,-2.49) node[rground]{};
        \draw (-4.05,-2.49) node[rground]{};
        \draw (-2,-2.49) node[rground]{};
        \draw (2.55,-2.49) node[rground]{};

        \draw (-4,0.49) -- (-4.5,0.49);
        \draw (-2,-0.49) -- (-1,-0.49);
        
        \draw (-2,0.49) -- (-1,0.49);
        \draw (-4,0.49) -- (-5.5,0.49);
        \draw (-1.5,0.49) -- (-1.5,2);
        \draw (-1.5,2) -- (-1,2);
        \draw (1,2) -- (1.5,2);
        \draw (1.5,2) -- (1.5,0);
        \draw (1,0) -- (2.5,0);

        \draw (-5.55,0.49) circle (1.5pt);
    	\draw (-5.55,-2.49) circle (1.5pt);
        \draw (-4.05,-0.49) circle (1.5pt);
        \draw (-4.05,-2.49) circle (1.5pt);
        \draw (2.55,0) circle (1.5pt);
    	\draw (2.55,-2.49) circle (1.5pt); 
        \draw[black,fill=black] (-1.5,0.49) circle (1.5pt);
    	\draw[black,fill=black] (1.5,0) circle (1.5pt);

        \draw[->, thick, blue] (-1.2,0.4) -- (-1.2,-0.4) node[above left, blue] {$U_D$};
        \draw[->, thick, blue] (-5.55,0.4) -- (-5.55,-2.4) node[above left, blue] {$U_{e_2}$};
        \draw[->, thick, blue] (-4.05,-0.6) -- (-4.05,-2.4) node[above left, blue] {$U_{e_1}$};
        \draw[->, thick, blue] (2.55,-0.1) -- (2.55,-2.4) node[above left, blue] {$U_a$};
\end{circuitikz}
\end{center}

\begin{align}
    U_a=(U_{e_2}-U_{e_1}\cdot{\frac{R_2}{R_1}})
\end{align}

\subsection{Typ 2}
\begin{center}
\begin{circuitikz}
        \draw node [op amp](A1){};
        \draw(-4,0.49) to[R, l=$R_2$] (-2,0.49);
        \draw(-4,2) to[R, l=$R_2$] (-2,2);
        \draw(-1,2) to[R, l=$R_4$] (1,2);
        \draw(-4,-0.49) to[R, l=$R_1$] (-2,-0.49);
        \draw(-4,-1.49) to[R, l=$R_1$] (-2,-1.49);
        \draw(-2,-1.49) to[R, l=$R_3$] (-2,-3.49);
        \draw (-2,-3.49) node[rground]{};
        \draw (2.55,-2.49) node[rground]{};

        
        \draw (-2,-0.49) -- (-1,-0.49);
        \draw (-2,0.49) -- (-1,0.49);
        \draw (-1.5,0.49) -- (-1.5,2);
        \draw (-1.5,2) -- (-1,2);
        \draw (1,2) -- (1.5,2);
        \draw (1.5,2) -- (1.5,0);
        \draw (1,0) -- (2.5,0);
        \draw (-2,-1.49) -- (-2,-0.49);
        \draw (-1.5,2) -- (-2,2);

        \draw (-4.05,0.49) circle (1.5pt);
    	\draw (-4.05,2) circle (1.5pt);
        \draw (-4.05,-0.49) circle (1.5pt);
        \draw (-4.05,-1.49) circle (1.5pt);
        \draw (2.55,0) circle (1.5pt);
    	\draw (2.55,-2.49) circle (1.5pt); 
        \draw[black,fill=black] (-1.5,0.49) circle (1.5pt);
    	\draw[black,fill=black] (1.5,0) circle (1.5pt);
        \draw[black,fill=black] (-1.5,2) circle (1.5pt);
        \draw[black,fill=black] (-2,-0.49) circle (1.5pt);
        \draw[black,fill=black] (-2,-1.49) circle (1.5pt);

        \draw[ thick, red] (-3.8,2.7) -- (-2.2,2.7) node[above left, red] {};
        \draw[ thick, red] (-2.2,2.7) -- (-2.2,0.2) node[above left, red] {};
        \draw[ thick, red] (-2.2,0.2) -- (-3.8,0.2) node[above left, red] {};
        \draw[ thick, red] (-3.8,0.2) -- (-3.8,2.7) node[above left, red] {$A$}; 

        \draw[ thick, green] (-3.8,0.15) -- (-2.2,0.15) node[above left, green] {};
        \draw[ thick, green] (-2.2,0.15) -- (-2.2,-2) node[above left, green] {};
        \draw[ thick, green] (-2.2,-2) -- (-3.8,-2) node[below left, green] {$B$};
        \draw[ thick, green] (-3.8,-2) -- (-3.8,0.15) node[above left, green] {}; 

        \draw[->, thick, blue] (-1.2,0.4) -- (-1.2,-0.4) node[above left, blue] {$U_D$};
        \draw[->, thick, blue] (2.55,-0.1) -- (2.55,-2.4) node[above left, blue] {$U_a$};
\end{circuitikz}
\end{center}

\begin{align}
    U_a=(\sum B - \sum A)\cdot\frac{R_2}{R_1}
\end{align}

\subsection{Typ 3}
\begin{center}
\begin{circuitikz}
        \draw node [op amp](A1){};
        \draw(-4,0.49) to[R, l=$R_2$] (-2,0.49);
        \draw(-1,2) to[R, l=$R_4$] (1,2);
        \draw(-4,-0.49) to[R, l=$R_1$] (-2,-0.49);
        \draw(-2,-0.49) to[R, l=$R_3$] (-2,-2.49);
        \draw (-5.55,-2.49) node[rground]{};
        \draw (-4.05,-2.49) node[rground]{};
        \draw (-2,-2.49) node[rground]{};
        \draw (2.55,-2.49) node[rground]{};

        \draw (-4,0.49) -- (-4.5,0.49);
        \draw (-2,-0.49) -- (-1,-0.49);
        
        \draw (-2,0.49) -- (-1,0.49);
        \draw (-4,0.49) -- (-5.5,0.49);
        \draw (-1.5,0.49) -- (-1.5,2);
        \draw (-1.5,2) -- (-1,2);
        \draw (1,2) -- (1.5,2);
        \draw (1.5,2) -- (1.5,0);
        \draw (1,0) -- (2.5,0);

        \draw (-5.55,0.49) circle (1.5pt);
    	\draw (-5.55,-2.49) circle (1.5pt);
        \draw (-4.05,-0.49) circle (1.5pt);
        \draw (-4.05,-2.49) circle (1.5pt);
        \draw (2.55,0) circle (1.5pt);
    	\draw (2.55,-2.49) circle (1.5pt); 
        \draw[black,fill=black] (-1.5,0.49) circle (1.5pt);
    	\draw[black,fill=black] (1.5,0) circle (1.5pt);

        \draw[->, thick, blue] (-1.2,0.4) -- (-1.2,-0.4) node[above left, blue] {$U_D$};
        \draw[->, thick, blue] (-5.55,0.4) -- (-5.55,-2.4) node[above left, blue] {$U_e2$};
        \draw[->, thick, blue] (-4.05,-0.6) -- (-4.05,-2.4) node[above left, blue] {$U_e1$};
        \draw[->, thick, blue] (2.55,-0.1) -- (2.55,-2.4) node[above left, blue] {$U_a$};
\end{circuitikz}
\end{center}

\underline{Berechnung mit Überlagerungsprinzip} \\
\underline{$U_{e_1}$ wirkt, $U_{e_2}=0$:}
\begin{align}
    U_a'=\frac{R_4+R_2}{R_2}\cdot U_{e_1}\cdot\frac{R_3}{R_1+R_3}
\end{align}

\underline{$U_{e_2}$ wirkt, $U_{e_2}=1$:}
\begin{align}
    U_a''=-U_{e_2}\cdot \frac{R_4}{R_2}
\end{align}

\underline{Gesamt}
\begin{align}
    U_a&=U_a'+U_a'' \\
    U_a&=U_{e_1}\cdot (\frac{R_4+R_2}{R_2}\cdot\frac{R_3}{R_1+R_3})-U_{e_2}\cdot\frac{R_4}{R_2}
\end{align}

Wenn alle $R$ gleich groß sind, gilt: $U_a=U_{e_1}-U_{e_2}$

\todo{Berechnung mit Teilspannungen}

\newpage

\section{Integrator}


\begin{center}
\begin{circuitikz}
        \draw node [op amp](A1){};
        \draw(-4,0.49) to[R, l=$R_1$] (-2,0.49);
        \draw(-1,2) to[R, l=$R_2$] (1,2);
        \draw(-1,3.5) to[C, l=$C$] (1,3.5);
        \draw (-4.55,-1.55) node[rground]{};
        \draw (-2,-1.5) node[rground]{};
        \draw (2.55,-1.55) node[rground]{};

        \draw (-4,0.49) -- (-4.5,0.49);
        \draw (-2,-0.49) -- (-1,-0.49);
        \draw (-2,-0.49) -- (-2,-1.5);
        \draw (-2,0.49) -- (-1,0.49);
        \draw (-1.5,0.49) -- (-1.5,2);
        \draw (-1.5,2) -- (-1,2);
        \draw (1,2) -- (1.5,2);
        \draw (1.5,2) -- (1.5,0);
        \draw (1,0) -- (2.5,0);
        \draw (-1.5,2) -- (-1.5,3.5);
        \draw (-1.5,3.5) -- (-1,3.5);
        \draw (1.5,3.5) -- (1,3.5);
        \draw (1.5,2) -- (1.5,3.5);

        \draw (-4.55,0.49) circle (1.5pt);
    	\draw (-4.55,-1.5) circle (1.5pt); 
        \draw (2.55,0) circle (1.5pt);
    	\draw (2.55,-1.5) circle (1.5pt); 
        \draw[black,fill=black] (-1.5,0.49) circle (1.5pt);
    	\draw[black,fill=black] (1.5,0) circle (1.5pt);

        \draw[->, thick, blue] (-1.2,0.4) -- (-1.2,-0.4) node[above left, blue] {$U_D$};
        \draw[->, thick, blue] (-4.55,0.4) -- (-4.55,-1.4) node[above left, blue] {$U_e$};
        \draw[->, thick, blue] (2.55,-0.1) -- (2.55,-1.4) node[above left, blue] {$U_a$};
\end{circuitikz}
\end{center}

\section{Differentiator}
\begin{center}
\begin{circuitikz}
        \draw node [op amp](A1){};
        \draw(-6,0.49) to[R, l=$R_1$] (-4,0.49);
        \draw(-1,2) to[R, l=$R_2$] (1,2);
        \draw(-4,0.49) to[C, l=$C_1$] (-2,0.49);
        \draw (-6.05,-1.55) node[rground]{};
        \draw (-2,-1.5) node[rground]{};
        \draw (2.55,-1.55) node[rground]{};

        
        \draw (-2,-0.49) -- (-1,-0.49);
        \draw (-2,-0.49) -- (-2,-1.5);
        \draw (-2,0.49) -- (-1,0.49);
        \draw (-1.5,0.49) -- (-1.5,2);
        \draw (-1.5,2) -- (-1,2);
        \draw (1,2) -- (1.5,2);
        \draw (1.5,2) -- (1.5,0);
        \draw (1,0) -- (2.5,0);

        \draw (-6.05,0.49) circle (1.5pt);
    	\draw (-6.05,-1.5) circle (1.5pt); 
        \draw (2.55,0) circle (1.5pt);
    	\draw (2.55,-1.5) circle (1.5pt); 
        \draw[black,fill=black] (-1.5,0.49) circle (1.5pt);
    	\draw[black,fill=black] (1.5,0) circle (1.5pt);

        \draw[->, thick, blue] (-1.2,0.4) -- (-1.2,-0.4) node[above left, blue] {$U_D$};
        \draw[->, thick, blue] (-6.05,0.4) -- (-6.05,-1.4) node[above left, blue] {$U_e$};
        \draw[->, thick, blue] (2.55,-0.1) -- (2.55,-1.4) node[above left, blue] {$U_a$};
        
        
\end{circuitikz}
\end{center}

\newpage

\section{Pegelwandler}
\begin{center}
\begin{circuitikz}
        \draw node [op amp,  noinv input up](A1){};
        \draw(-4,0.49) to[R, l=$R_1$] (-2,0.49);
        \draw(-1,2) to[R, l=$R_2$] (1,2);
        \draw (-4.55,-1.55) node[rground]{};
        \draw (-2,-1.5) node[rground]{};
        \draw (2.55,-1.55) node[rground]{};

        \draw (-4,0.49) -- (-4.5,0.49);
        \draw (-2,-0.49) -- (-1,-0.49);
        \draw (-2,-0.49) -- (-2,-1.5);
        \draw (-2,0.49) -- (-1,0.49);
        \draw (-1.5,0.49) -- (-1.5,2);
        \draw (-1.5,2) -- (-1,2);
        \draw (1,2) -- (1.5,2);
        \draw (1.5,2) -- (1.5,0);
        \draw (1,0) -- (2.5,0);

        \draw (-4.55,0.49) circle (1.5pt);
    	\draw (-4.55,-1.5) circle (1.5pt); 
        \draw (2.55,0) circle (1.5pt);
    	\draw (2.55,-1.5) circle (1.5pt); 
        \draw[black,fill=black] (-1.5,0.49) circle (1.5pt);
    	\draw[black,fill=black] (1.5,0) circle (1.5pt);

        \draw[->, thick, blue] (-1.2,0.4) -- (-1.2,-0.4) node[above left, blue] {$U_D$};
        \draw[->, thick, blue] (-4.55,0.4) -- (-4.55,-1.4) node[above left, blue] {$U_e$};
        \draw[->, thick, blue] (2.55,-0.1) -- (2.55,-1.4) node[above left, blue] {$U_a$};
        \draw[->, thick, blue] (-4.55,3.5) -- (1.5,3.5) node[above left, blue] {$U_{abfall}$};
        \draw[->, thick, blue] (-1.2,2.7) -- (1.5,2.7) node[above left, blue] {$U_{2}$};
\end{circuitikz}
\end{center}
Hier gilt:
\begin{align}
    V=-\frac{R_2}{R_1}=\frac{\Delta U_a}{\Delta U_e}
\end{align}

\subsubsection*{Beispiel}
Das Eingangssignal von $U_e=-1V$ bis $+1V$ soll am Ausgang zu $U_a=0V$ bis $+5V$ gewandelt werden.
\begin{align}
    &V=-\frac{R_2}{R_1}=\frac{\Delta U_a}{\Delta U_e}   \\
    &V=-\frac{5V-0V}{1V-(-1V)}=\frac{5V}{2V}            \\
    &V=-\frac{5k\Omega}{2k\Omega}
\end{align}

\begin{align}
    \frac{U_{R_2}}{U_e-U_a}&=\frac{R_2}{R_1+R_2}                \\
    U_{R_2}&=(U_e-U_a)-\frac{R_2}{R_1+R_2}                      \\
    U_{R_2}&=(1V-0V)-\frac{5k\Omega}{2k\Omega+5k\Omega}         \\
    U_{R_2}&\approx 0,715V                                      \\
    \Rightarrow U_+=U_a+U_{R_2}&=0V+0,714V=0,714V
\end{align}

\newpage

\section{Instrumentation-Amplifier}
\begin{center}
\begin{circuitikz}
        \draw node [op amp](A1) at (0,-3) {}  ;
        \draw node [op amp,  noinv input up]() at (0,5){};
        \draw node [op amp](A1) at (6,1){};
        \draw (1.2,-1) to[vR, l=$R_{gain}$] ++(0,4);
        \draw(1.2,5) to[R, l=$R_2$] (1.2,3);
        \draw(1.2,-1) to[R, l=$R_2$] (1.2,-3);
        \draw(1.2,5) to[R, l=$R_3$] (3.2,5);
        \draw(1.2,-3) to[R, l=$R_3$] (3.2,-3);
        \draw(4.8,5) to[R, l=$R_4$] (7.2,5);
        \draw(4.8,-3) to[R, l=$R_4$] (7.2,-3);
        
        \draw (-2.05,3) node[rground]{};
        \draw (-2.05,-5) node[rground]{};
        \draw (7.2,-3) node[rground]{};
        \draw (8.55,-3) node[rground]{};

        \draw (-1.2,4.5) -- (-1.2,3);
        \draw (-1.2,3) -- (1.2,3);
        \draw (-1.2,-2.5) -- (-1.2,-1);
        \draw (-1.2,-1) -- (1.2,-1);
        \draw (3,5) -- (4.8,5);
        \draw (4.8,5) -- (4.8,1.5);
        \draw (3,-3) -- (4.8,-3);
        \draw (4.8,-3) -- (4.8,0.5);
        \draw (7.2,5) -- (7.2,1);
        \draw (7.2,1) -- (8.5,1);
        \draw (-2,5.5) -- (-1,5.5);
        \draw (-2,-3.5) -- (-1,-3.5);


        \draw (-2.05,5.5) circle (1.5pt);
        \draw (-2.05,3.05) circle (1.5pt); 
        \draw (-2.05,-3.5) circle (1.5pt);
    	\draw (-2.05,-4.95) circle (1.5pt);
    	\draw (8.55,1) circle (1.5pt);
        \draw (8.55,-2.95) circle (1.5pt);
     
        \draw[black,fill=black] (7.2,1) circle (1.5pt);
    	\draw[black,fill=black] (1.2,5) circle (1.5pt);
        \draw[black,fill=black] (1.2,3) circle (1.5pt);
    	
        \draw[black,fill=black] (1.2,-1) circle (1.5pt);
    	\draw[black,fill=black] (4.8,5) circle (1.5pt);
        \draw[black,fill=black] (4.8,-3) circle (1.5pt);
    	\draw[black,fill=black] (1.2,-3) circle (1.5pt);
     
        \draw[->, thick, blue] (-2.05,5.4) -- (-2.05,3.15) node[above left, blue] {$U_1$};
        \draw[->, thick, blue] (-2.05,-3.6) -- (-2.05,-4.85) node[above left, blue] {$U_2$};
        \draw[->, thick, blue] (8.55,0.9) -- (8.55,-2.85) node[above left, blue] {$U_a$};
\end{circuitikz}
\end{center}