\chapter{Kirchhoff}
Kirchhoff hat zwei fundamentale Regeln bzw. Gesetze aufgestellt: \\
Die \textbf{Knotenregel} und die \textbf{Maschenregel}.

\section{Knotenregel}
Die Summe aller Ströme bei einem Knotenpunkt ist $0$, d.h. Ströme die hineinfließen, müssen auch hinausfließen:
\begin{align}
    \sum_{i = 0}^{N} I_i = 0
\end{align}

Diese Regel besagt, dass alle Ströme in einer Serienschaltung die gleichen sein müssen. Ebenso müssen Ströme die zu einem zusammenlaufen, aufsummiert werden, um den Ausgangsstrom zu ermitteln.

\section{Maschenregel}
Die Summer aller Spannungen in einer Masche ist $0$.

\begin{align}
    \sum_{i = 0}^{N} U_i = 0
\end{align}

Laut dieser Regel sind Spannungen einer Parallelschaltung immer gleich groß. Auch gilt, dass die Summe von Spannungen einer Serienschaltungen aufsummiert werden müssen, um die Gesamtspannung zu berechnen.