\chapter{Leistung}
\begin{itemize}
    \item \underline{Wirkleistung $P$} \\
    Tatsächlich umgesetzte Energie in Watt [W]
    \item \underline{Blindleistung $Q$} \\
    Unerwünschte bzw. nicht nutzbare Energie in Volt-Ampere Relativ [var]
    \item \underline{Scheinleistung $S$} \\
    Gesamtleistung in Volt-Ampere [VA]
\end{itemize}

\section{Leistung bei Gleichstrom}
Allgemein gilt:

\begin{align}
    P&=U\cdot I \\
    P&=\frac{U^2}{R} \\
    P&=I^2\cdot R    
\end{align}

\begin{itemize}
    \item $P$ ... Leistung
    \item $U$ ... Spannung
    \item $I$ ... Strom
    \item $R$ ... Widerstand
\end{itemize}

\newpage

Ebenso: \\
\underline{Wirkleistung $P$}
\begin{align}
    P = U_w \cdot I = U \cdot I \cdot cos(\varphi)
\end{align}
\begin{itemize}
    \item $P$ ... Wirkleistung in [W]
    \item $U_w$ ... Wirkkomponente der Spannung in [V]
\end{itemize}

\underline{Blindleistung $Q$}
\begin{align}
    Q = U_b \cdot I = U \cdot I \cdot sin(\varphi)
\end{align}
\begin{itemize}
    \item $Q$ ... Blindleistung in [var]
    \item $U_b$ ... Blindkomponente der Spannung in [V]
\end{itemize}
Die induktive Blindleistung ist positiv: $sin(\varphi) > 0 \rightarrow Q_L > 0$
Die kapazitive Blindleistung ist negativ: $sin(\varphi) < 0 \rightarrow Q_C < 0$

\underline{Scheinleistung $S$}
\begin{align}
    S = U \cdot I
\end{align}
\begin{itemize}
    \item $S$ ... Scheinleistung in [VA]
\end{itemize}

\newpage

\section{Leistung bei Wechselstrom}
Bei einem sinusförmigen Verlauf der Spannung $u$ und des Stroms $i$ gelten folgende Gleichungen:
\begin{align}
    u = \hat{U} \cdot sin(\omega \cdot t) \hspace{1cm} i = \hat{I} \cdot sin(\omega \cdot t - \varphi)
\end{align}
Werden die Momentanwerte von $u$ und $i$ miteinander multipliziert, erhält man den Momentanwert der Leistung $p$:
\begin{align}
    p = u \cdot i = \hat{U} \cdot sin(\omega \cdot t) \cdot \hat{I} \cdot sin(\omega \cdot t - \varphi)
\end{align}
oder:
\begin{align}
    p = U_{eff} \cdot I_{eff} \cdot cos(\varphi) - U_{eff} \cdot I_{eff} \cdot (2\cdot \omega \cdot t - \varphi)
\end{align}

\begin{itemize}
    \item $p$ ... Momentanwert der Wechselstromleistung in [W]
    \item $U$ ... Effektivwert der Spannung in [V]
    \item $I$ ... Effektivwert des Stromes in [A]
\end{itemize}

\underline{Leistsungsfaktor} \\
Der Leistungsfaktor $cos(\varphi)$ gibt an, welchen Anteil die Wirkleistung an der Scheinleistung hat. Er erreicht bei ohmschen Lasten den maximalen Wert $1$:
\begin{align}
    P &= U \cdot I \cdot cos(\varphi) = S \cdot cos(\varphi) \\
    cos(\varphi) &= \frac{P}{S} 
\end{align}
\begin{itemize}
    \item $P$ ... Wirkleistung in [W]
    \item $S$ ... Scheinleistung in [VA]
    \item $cos(\varphi)$ ... Leistungsfaktor
\end{itemize}

\subsection{Kompensation}