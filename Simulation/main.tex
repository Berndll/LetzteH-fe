\chapter{Simulation}

\section{Altium}
Es muss beachtet werden, dass nicht alle BAuteile simmuliert werden können, sondern nur die der "Simulation Generic Components" Library.\\

Außerdem ist es besonders wichtig, dass bei \textbf{OPV-Schaltungen} die einen \textbf{Verstärker} implementieren, der \textbf{OPV} benutzt werden muss.\\
Bei Schaltungen die den OPV als \textbf{Schmitttrigger} (oder etwas Ähnliches) verwenden, muss der Komperator verwendet werden.

\subsection{Quellen}
Ab AltiumDesigner21 sind alle Quellen gleich; unter \verb|Simulate| $\rightarrow$ \verb|Sources| können Quellen platziert werden:

\subsection{Probes}
Es können unter \verb|Simulate| $\rightarrow$ \verb|Place Probes| Probes platziert werden, mit denen Spannungen und Ströme gemessen werden können. In AD17 war bzw. ist dies mit Netlabels möglich.

\subsection{Simulation}
Unter \verb|Simulate| $\rightarrow$ \verb|Simulation Dashboard| kann die Simulation eingestellt und gestartet werden. \\
Zunächst wird die Schaltung verifiziert; um die Verifikation zu aktualisieren, muss im Drop-Down-Menü ein anderes Dokument ausgewählt werden und dann zurückgewechselt werden. \\

Danach werden die Quellen und Probes ausgewählt: Mit einem Häkchen können diese entsprechend deaktiviert werden. Mit dem gefärbten Kästchen kann die Farbe der Probe im Plot verändert werden. \\

Als Nächstes wird die Simulation eingestellt, wobei \verb|Transient| und \verb|AC-Sweep| besonders wichtig sind:
\begin{itemize}
    \item \underline{Transient-Analyse} \\
    Bei der Transienten-Analyse wird die Schaltung in einem bestimmen Zeitbereich simuliert. Dieser kan sowohl in Zeit, als auch in Perioden angegeben werden (wobei diese nur bei Quellen mit periodischen Signalen funktioniert). Die zuvor ausgewählten Probes werden automatisch in einem Plot angezeigt. Mit dem "\verb|+ Add|"-Knopf können weiter Plots bzw. Signale hinzugefügt werden (siehe \todo{Add Reference}). \\
    Mit dem Häkchen "\verb|Use Initial Conditions|" können Anfangsspannungen festgelegt werden, welche in der Schematik über \verb|Simulate| $\rightarrow$ \verb|Place Initial Condition| platziert werden. \\
    In den Einstellungen der \verb|Initial Condition| (erreichbar via Doppelklick) kann unter \verb|Parameters| die Spannung eingstellt werden.

    \item \underline{AC-Sweep} \\
    Beim AC-Sweep wird die Frequenz der angegebenen Quellen in einem festgelegten Bereich "gesweeped", d.h. vom Minimum zum Maximum schrittweise durchgerechnet. Die Frequenz kann in Dekaden (logarithmisch), Okataven oder Linear angegeben werden. \\

    Mit dem "\verb|+ Add|"-Knopf können zusätzliche Outputs hinzugefügt werden, wodurch ein neues Feld erscheint. In diesem kann über den Drei-Punkte-Knopf der Output konfiguriert werden.\\

    In \verb|Waveforms| stehen alle möglichen Signale, welche in der Schaltung vorkommen (Probes, Netlabels, Widerstandsspannungen, etc.) Probes erhalten den Namen folgendermaßen: \verb|v("Net", Bauteil, "_", Probe Nummer)|. Eine Probe mit der Nummer $1$, an einem Widerstand, würde dementsprechend \verb|v(NetR1_1)| heißen. \\

    In \verb|Functions| stehen alle möglichen Operationen wie Addition, Umrechnen in dB oder Berechnung der Phase. In Expression-X/Y steht was wirklich angezeigt werden soll. \\
    Beachte dass es das Expression-X-Feld in AD21 nicht gibt.

    \subsubsection*{Beispiel}
    Bodediagramm eines Tiefpasses.

    \begin{itemize}
        \item \underline{Übertragungsfunktion}: Expression-Y: \verb|db(v(NetC1_2/v(NetR1_1)))|
        \item \underline{Phasengang}: Expression-Y: \verb|PHASE(v(NetC1_2))|
    \end{itemize}
    \verb|Net1| ist der Eingang des Tiefpasses, \verb|Net2| der Ausgang.

    \underline{Simulation Dashboard}
    \underline{Simulation Output}

    Sollte eine Achse fehlen: doppelklicken auf die Achse und unter "Label" die gewünschte Einheit eintragen. \\

    Die entsprechende Schematik des Beispiels:
\end{itemize}

\newpage

\section{MicroCap}
\subsection{Komponentenauswahl}
\begin{figure}[h]
    \centering
    % \includegraphics[width=0.7\linewidth]{Microcap-Bilder/Bauteileaussuchen.PNG}
    \caption{Komponentenauswahl}
    \label{fig:Komponentenauswahl}
\end{figure}

\subsubsection*{Komponenten suchen}
$\rightarrow$ Components $\rightarrow$ Find Component oder:\\
in linkem Fenster $\rightarrow$ Search
\begin{figure}[h]
    \centering
    % \includegraphics[width=0.7\linewidth]{Microcap-Bilder/Bauteilesuchen.PNG}
    \caption{Komponenten suchen}
    \label{fig:Komponenten suchen}
\end{figure}

\newpage

\subsection{Bauteile verbinden}
$\rightarrow$ Wire Mode $\rightarrow$ linke Maustaste gedrückt halten um Verbindungen zu ziehen
\begin{figure}[h]
    \centering
    % \includegraphics[width=1\linewidth]{Microcap-Bilder/Bauteileverbinden.PNG}
    \caption{Bauteile verbinden}
    \label{fig:Bauteile verbinden}
\end{figure}

\subsection{Bauteile konfigurieren}
jedes Bauteil muss individuell konfiguriert werden $\rightarrow$ auf Syntax achten wenn unklar\\
mit "Plot" können Diagramme angezeigt werden (z.Bsp. bei Pulse Source das eingestellte Signal im Zeitbereich, beim Kondensator die Impedanz im Vergleich zur Frequenz, ...)

\subsubsection*{Werte}
Microcap unterscheidet nicht zwischen Groß-und Kleinschreibung!
$\rightarrow$ 1M = 1m = 1 milli
\begin{itemize}
    \item Mega $\rightarrow$ meg
    \item Kilo $\rightarrow$ k
    \item Milli $\rightarrow$ m
    \item µ $\rightarrow$ u
    \item Nano $\rightarrow$ n
\end{itemize}

\subsection*{Sinussignal}
\begin{itemize}
    \item MODEL: GENERAL
    \item A: Amplitude[V]
    \item F: Frequenz[Hz]
\end{itemize}

\subsection*{Rechtecksignal}
\begin{itemize}
    \item MODEL: SQUARE
    \item VONE: "HIGH"
    \item VZERO: "LOW"
    \item Px: $\rightarrow$ Abb. \ref{fig:Konfiguration Rechtecksignal}
    \item Überprüfen: Plot
\end{itemize}

\begin{figure}[h]
    \centering
    % \includegraphics[width=1\linewidth]{Microcap-Bilder/Rechtecksignal.PNG}
    \caption{Konfiguration Rechtecksignal}
    \label{fig:Konfiguration Rechtecksignal}
\end{figure}

\newpage

\subsection*{OPV}

\subsubsection*{Schmitt-Trigger}
Für einen Schmitt-Trigger mit unsymmetrischer Versorgung (0V, 5V), der ungefähr zwischen 0V und 4,5V hin-und herschaltet (vorausgesetzt er wurde richtig beschalten ;P), müssen folgende Einstellungen verwendet werden. (siehe Abb.)

\begin{itemize}
    \item MODEL: \$GENERIC
    \item VNS: 0.6
    \item VSS: 5
    \item VEE: 0
    \item VPS: 4
\end{itemize}

\begin{figure}[h]
    \centering
    % \includegraphics[width=0.65\linewidth]{Microcap-Bilder/invertierenderSchmitttrigger.PNG}
    \caption{Einstellungen für Schmitt-trigger}
    \label{fig:Einstellungen für Schmitt-trigger}
\end{figure}

\subsubsection*{Power Supplies}
Platziert man einen OPV, werden manchmal von Microcap automatisch Labels hinzugefügt, die den OPV mit 15V und -15V versorgen.\\
Diese Labels kann man löschen und durch eine andere Versorgung ersetzen (siehe Fixed Analog) oder man verändert die Werte unter Power Supplies:

\begin{figure}[h]
    \centering
    % \includegraphics[width=1\linewidth]{Microcap-Bilder/PowerSupplies.PNG}
    \caption{Power Supplies}
    \label{fig:Power Supplies}
\end{figure}

\subsubsection*{Transformator}

\begin{itemize}
    \item VALUE: Primärspule, Sekundärspule, Kopplungsfaktor\\
    z.Bsp: VALUE = 120u, 1m, 0.1
    \item der Pin "Plus/Minus output" ist die Primärspule und der Pin "Plus/Minus Input" ist die Sekundärspule.
\end{itemize}

\subsection{Fixed Analog - Spannungsversorgung}
Dieses Bauteil liefert eine fixe Gleichspannung.\\
Dadurch kann man sich mehrere Verbindungen, die beispielsweise bei einer Spannungsquelle nötig wären, ersparen.\\
(Findet man indem man danach sucht (siehe \textit{Komponenten suchen}))

\newpage

\subsection{Simulationspunkte}
$\rightarrow$ Text Mode $\rightarrow$ Text platzieren und passenden Namen geben (z. Bsp: U...) $\rightarrow$ mit Leitung verbinden (erkennt man am roten Punkt)

\begin{figure}[h]
    \centering
    % \includegraphics[width=0.5\linewidth]{Textmode.PNG}
    \caption{Text Mode}
    \label{fig:Text Mode}
\end{figure}

\newpage

\subsection{Simulation}

\subsubsection*{Zeitbereich}
\textbf{Analysis $\rightarrow$ Transient}

\begin{itemize}
    \item Maximum Run Time: bis zu diesem Zeitpunkt wird simuliert
    \item Output Start Time (tstart): Zeitpunkt, an der die Simulation beginnt
    \item Maximum Time Step: $\rightarrow$ 0
    \item Number of Points: Bestimmt die Anzahl der simulierten Punkte $\rightarrow$ je höher, desto länger dauert die Simulation
    \item Temperature: Temperatur
\end{itemize}

\begin{figure}[h]
    \centering
    % \includegraphics[width=1\linewidth]{Microcap-Bilder/Transientenanalyse-Diagramme.PNG}
    \caption{Transientenanalyse - Einstellungen der Graphen und Achsen}
    \label{fig:Transientenanalyse - Einstellungen der Graphen und Achsen}
\end{figure}

\textbf{Abb. \ref{fig:Transientenanalyse - Einstellungen der Graphen und Achsen} von links nach rechts:}
\begin{itemize}
    \item \begin{itemize}
        \item Grün: Graph wird angezeigt
        \item Gelb: Graph wird simuliert, aber nicht angezeigt
        \item Rot: Graph wird weder simuliert noch angezeigt
    \end{itemize}
    \item \begin{itemize}
        \item Grün: X-Achse linear skaliert
        \item Blau: X-Achse logarithmisch skaliert
    \end{itemize}
    \item \begin{itemize}
        \item Grün: Y-Achse linear skaliert
        \item Blau: Y-Achse logarithmisch skaliert
    \end{itemize}
    \item Farbenauswahl für die Graphen
    \item Page: Auf welcher Seite der Graph ist (1, 2, ...)
    \item P: In welchem Diagramm der Graph ist (1, 2, ...)
    \item X Expression: T $\rightarrow$ Zeit
    \item Y Expression: \begin{itemize}
        \item v(Simulationspunkt) - Spannung (z. Bsp. v(Uin))
        \item I(...) - Strom (z. Bsp. I(C1))
        \item normale Rechenoperationen können verwendet werden (z.Bsp. v(Uin)-v(Uout))
    \end{itemize}
    \item X-Range: Der Bereich der x-Achse, wobei Auto die einfachste Variante ist; TMAX ist die Maximalzeit. Ansonsten können hier auch - mit Beistrichen getrennt - der Endwert, Startwert und der Unterteilungsabstand festgelegt werden.
    \item Y-Range: gleiche Einstellung wie bei X-Range
\end{itemize}

\subsubsection*{Bodediagramm}

\begin{itemize}
    \item Frequency Range: letzter simulierter Frequenzpunkt, erster simulierter Frequenzpunkt (z. Bsp: 1meg, 10k)
    \item X-Achsen: logarithmisch skalieren
    \item X-Expression: F $\rightarrow$ Frequenz
    \item Y-Expression: \begin{itemize}
        \item Betrag: z. Bsp. dB(v(Uout))
        \item Phase: z. Bsp. ph(v(Uout))
    \end{itemize}
\end{itemize}

\begin{figure}[h]
    \centering
    % \includegraphics[width=1\linewidth]{Microcap-Bilder/ACAnalyse-Limits.PNG}
    \caption{AC-Analyse $\rightarrow$ Bodediagramm}
    \label{fig:Bodediagramm}
\end{figure}
% Folgende Schritte müssen befolgt werden, um in \textbf{Microcap} simulieren zu können:
% \begin{enumerate}
%     % \item Zunächst muss eine Datei erstellt werden: \\
%     % \verb|File| $\rightarrow$ \verb|New| $\rightarrow$ \verb|Schematic File (.cir)| $\rightarrow$ \verb|OK|

%     \item \underline{Komponentenauswahl} \\
%     \begin{itemize}
%         \item Allgemeine Bauteile (Widerstände, Kondensatoren, Quellen, etc.): \verb|Component| $\rightarrow$ \verb|Analog Primitives| $\rightarrow$ \verb|Benötigtes Bauteil|
%         \item Spezifische Bauteile (Transistoren, OPVs, etc.): \verb|Component| $\rightarrow$ \verb|Find Component| $\rightarrow$ \verb|Eingabe des Bauteilnamens| $\rightarrow$ \verb|Auswählen des Bauteils| $\rightarrow$ \verb|OK|
%     \end{itemize}
%     \item \underline{Verbinden der Bauteile} \\
%     Anklicken von Wire Mode. \todo{Add pic}

%     \item \underline{Konfiguration der Bauteile} \\
%     \textbf{Beispiel}
%     Pulse Source

%     \item \underline{Platzierung von Simulationspunkten} \\
%     Anklicken von \verb|Text Mode| und gewünschten Namen eingeben, welcher in der Simulation ausgewählt werden kann. Danach auf die gewünschte Stelle in der Schaltung ziehen, an der dann ein roter Punkt erscheinen sollte.

%     \item \underline{Einstellen der Simulation} \\
%     \begin{itemize}
%         \item \textbf{Im Zeitbereich} \\
%         \verb|Analysis| $\rightarrow$ \verb|Transient|
%         \begin{itemize}
%             \item \underline{Maximum Run Time}: Gesamtzeit der Simulation.
%             \item \underline{Number of Points}: Die Anzahl der simulierten Punkte.
%             \item \underline{Page}: Die Seite auf der der simulierte Graph gezeigt wird.
%             \item \underline{P}: Jede Seite kann mehrere Plots enthalten ($\approx$Graph); ist dies der Fall, werden sie untereinander dargestellt.
%             \item \underline{X Expression}: Wird im Normalfall mit $t$ als Zeit beschrieben, kann aber jedoch für etwas anderes verwendet werden.
%             \item \underline{Y Expression}: Dort sollten die jeweiligen Simulationspunkte ausgewählt werden, z.B.Spannungsverhältnisse, deren dB-Werte oder andere Funktionen bzw. Funktionswerte.
%             \item \underline{X Range}: Der Bereich der x-Achse, wobei \verb|Auto| die einfachste Variante ist; \verb|TMAX| ist die Maximalzeit.\\
%             Ansonsten können hier auch - mit Beistrichen getrennt - der Endwert, Startwert und der Unterteilungsabstand festgelegt werden.
%             \item \underline{Y Range}: Der Bereich der y-Achse; Optionen sind hier die selben wie bei \verb|X Range|.
%             \todo{Verschiedene Darstellungsmethoden hinzufügen.}
%         \end{itemize}

%     \end{itemize}
% \end{enumerate}
% \todo{Wirklich? Wie man eine Datei macht??}